\label{chap:background}
% Logic Language
\section{Logic Language}
Logic languages are a family of programming languages which express computation as proof search.
This makes them well suited to the task at hand because it allows us to describe the provenance of information and to separate execution strategy for a suite of analyses from the expression of said analyses.
I will review Datalog and Prolog as a way of giving a general overview of logic programming, but the field is quite broad.
%%Sentence omitted because it forward references too many concepts
%Datalog and Prolog are selected to give an example of both a forwards-chaining and backwards-chaining search, and to motivate the notion of a database of facts as our incremental store.
% * Datalog
\subsection{Datalog}
\subsubsection{Basics}
%   * Predicates
Predicates form the concept of a relation between several values.
For example,
\[
        \pred{parent}{\cdot, \cdot}
\]
could be used to represent a relation in which the left argument is the parent of the right argument.
%   * Facts
The concept of a ``fact'' is used to represent a particular member of a relation. For example,
\[
        \pred{parent}{\id{alice}, \id{bob}}
\]
represents the fact that $\id{alice}$ is $\id{bob}$'s parent.
The set of facts known is sometimes referred to as the database, and is separated into extensional and intensional parts.
The extensional portion of the database is those facts which are given to the system prior to execution, while the intensional database consists of those derived from the extensional database.
%   * Rules
In order to actually perform computation with this system and create such an extensional database, logic languages use ``rules'', which function very much like inference rules from traditional proof writing.
Following in the vein of the previous examples,
\[
        \frule{\pred{sibling}{\var{X}, \var{Y}}}{\pred{parent}{\var{P}, \var{X}}, \pred{parent}{\var{P}, \var{Y}}}
\]
expresses that if for some assignment to $\var{P}$, $\var{X}$, and $\var{Y}$, we know that $\var{X}$ shares parent $\var{P}$ with $\var{Y}$, then we can derive that $\var{X}$ and $\var{Y}$ are siblings.
The left hand side is referred to as the \emph{head clause} and represents the template from which new facts will be built by this rule.
The right hand side is called the \emph{body clause} and describes the search or match to be performed on the database.
Rules are applied ``to saturation'', meaning that they are applied repeatedly until the body clause no longer has any new matches in the database.
The order in which rules are declared is irrelevant: all rules are examined for matching body clauses repeatedly until none match.
%   * Simple examples
\subsubsection{Example: Uncle}
If we ignore the genderedness of the term ``Uncle'', calculating from a series of parent/child relations all of the uncles is a fairly straightforward example.
We begin with an extensional database consisting of:
\[
        \pred{parent}{\id{alice}, \id{bob}}
        \quad \pred{parent}{\id{alice}, \id{charlie}}
        \quad \pred{parent}{\id{bob}, \id{deb}}
        \quad \pred{parent}{\id{bob}, \id{ed}}
\]
We take the sibling rule from the earlier example, and also
\[
        \frule{\pred{uncle}{\var{X}, \var{Y}}}{\pred{sibling}{\var{X}, \var{P}}, \pred{parent}{\var{P}, \var{Y}}}
\]
which indicates that someone is considered an uncle if they are that person's parent's sibling.

With the rules and extensional database in hand, we can compute the complete database.
The uncle rule cannot apply, since there are no facts for the sibling predicate.
So, first applying the sibling rule to saturation yields a database of
\begin{gather*}
        \pred{parent}{\id{alice}, \id{bob}}
        \quad \pred{parent}{\id{alice}, \id{charlie}}
        \quad \pred{parent}{\id{bob}, \id{deb}}
        \quad \pred{parent}{\id{bob}, \id{ed}}\\
        \pred{sibling}{\id{bob}, \id{charlie}}
        \quad \pred{sibling}{\id{charlie}, \id{bob}}
        \quad \pred{sibling}{\id{deb}, \id{ed}}
        \quad \pred{sibling}{\id{ed}, \id{deb}}
\end{gather*}
The right hand side of the sibling rule no longer applies, so the only option now is to execute the uncle rule.
Doing so to saturation yields
\begin{gather*}
        \pred{parent}{\id{alice}, \id{bob}}
        \quad \pred{parent}{\id{alice}, \id{charlie}}
        \quad \pred{parent}{\id{bob}, \id{deb}}
        \quad \pred{parent}{\id{bob}, \id{ed}}\\
        \pred{sibling}{\id{bob}, \id{charlie}}
        \quad \pred{sibling}{\id{charlie}, \id{bob}}
        \quad \pred{sibling}{\id{deb}, \id{ed}}
        \quad \pred{sibling}{\id{ed}, \id{deb}}\\
        \pred{uncle}{\id{charlie}, \id{deb}}
        \quad \pred{uncle}{\id{charlie}, \id{ed}}
\end{gather*}
At this point, neither rule applies, and so the program has finished.
It has also derived the only two uncle facts possible.
\subsubsection{Example: Related}
Alternatively, perhaps we only wish to know if two people are related.
This is potentially more interesting because it is an obviously recursive search, unlike the previous example which was clearly considerable as a two-step search: find the siblings, then find siblings of parents.
However, since Datalog applies its rules until saturation, this is not any more difficult than the previous qurery.
Consider the rules
\begin{gather*}
        \frule{\pred{related}{\var{X}, \var{Y}}}{\pred{parent}{\var{X}, \var{Y}}}\\
        \frule{\pred{related}{\var{Y}, \var{X}}}{\pred{related}{\var{X}, \var{Y}}}\\
        \frule{\pred{related}{\var{X}, \var{Z}}}{\pred{related}{\var{X}, \var{Y}}, \pred{related}{\var{Y}, \var{Z}}}
\end{gather*}
The first rule translates a parent/child relationship into being related.
The second provides symmetry.
The third provides transitivity.
This is sufficient to compute the actual related set.
Execution proceeds much the same as the previous example, simply firing each rule until no rules are able to fire.
The only real difference is that the second and third rule will be fired interleaved due to the nature of the search.
However, as the intensional database is quite large for any nontrivial extensional database with these rules, the resulting database is omitted.
%   * Termination
\subsubsection{Termination}
It might be surprising to realize that despite recursive rules being permitted, the language as described thus far is actually terminating.
Head clauses construct facts with variables or directly with values.
If constructed with a variable, the variable must have been bound in the body clause.
If the variable was bound in the body clause, its value must have been in the database.
As a result, the only possible values are those which were in the extensional database and those mentioned concretely in the rules.
With a finite domain of values, and a finite list of predicates, there is a finite list of facts which could be instantiated.
Structure evaluation as a series of steps, each of which fires the first rule it can find which matches, or if none match, terminate.
Since this evaluation strategy is always productive or terminating, and we have established a finite upper bound on the database, we have a bound on the length of such an evaluation.
Since this evaluation strategy only produces facts according to the rules, and only stops when no rules apply, it matches the saturation/fixpoint semantics of Datalog.

While termination is a convenient property for a language to have, there are several caveats.
%
The upper bound is not a very practical one.
It is finite, which is sufficient for the proof, but will be very large even for simple quickly terminating programs.
The upper bound on the Uncle example from earlier would be $3 \times 5^2 = 75$ (5 database values, max arity of 2, 3 predicates).
This overapproximation will only get worse as new values or increased arity predicates appear.
%
Termination directly implies a limit to the language's power.
While a number of different kind of closure or search procedures are specifiable as Datalog programs, its inability to create new symbols is fairly limiting.
%
The proof is quite fragile, since many language modifications will introduce the ability to generate new symbols.
\subsubsection{External Predicates}
\label{sec:extpred}
%   * Motivation for external predicates (e.g. +2)
One example of a case where introducing new symbols might be desirable is the ability to perform arithmetic.
It is possible to encode arithmetic over some finite domain (e.g.\ $0-(2^{64} - 1)$) by adding a predicate for each binary operation and prepopulating the extensional database with all the values.
However, having such a table is massively impractical.

%   * External predicates
One way to deal with this issue is to introduce external predicates: predicates whose extents are not stored in the database, but are rather computed or queried from some external system when needed.
This leads to the need for a mode system.
Since I do not intend to use moded predicates in \sysname\, I will only give a general picture of the problem they solve.
Consider an external predicate $\pred{sum}{\cdot, \cdot, \cdot}$, where the first entry is the sum of the other two.
When trying to match the body clause of a rule, it becomes important that at least two entries be provided, or the resulting match set will have the enormous size of $|\mathbf{D}|^2$ where $\mathbf{D}$ is our domain set.
In the case of functions which are designed to be hard to invert (e.g.\ a hash function), this becomes not only a result size concern and an actual implementability concern.
To deal with this, external predicates can have mode signatures, explaining which entries are logical inputs, and which are outputs. For example, we could have
\[
        \pred{sum}{+, -, -} \quad \pred{hash}{+, -}
\]
A mode checking phase on the rule can then guarantee that matches against the sum predicate can be performed using
\[
        \textrm{add} : \textrm{uint64} \rightarrow \textrm{uint64} \rightarrow \textrm{uint64}
\]
rather than needing to assume the presence of subtraction and possible-addends operations.

%   * External predicates can break termination
Unfortunately, adding external predicates adds all possibilities for their result values to the upper bound for the termination.
Adding an external predicate which only produces a small type such as a boolean has little effect on our ability to bound execution.
Adding a large, but still finite bounded type like $\textrm{uint64}$ makes any numerical approximation of runtime from value count irrelevant, but still technicaly gives termination.
Adding a type with non-finite size like lists or strings \emph{breaks termination} despite these features being commonly desired for good reasons.

%   * Stratified negation
\subsubsection{Stratified Negation}
Another thing that can be useful to talk about is the inability of the system to prove a particular fact.
For example, this could be useful to describe routing.
Network reachability can be described similar to $\pred{related}{\cdot, \cdot}$
\begin{gather*}
        \frule{\pred{reachable}{\var{X}, \var{Y}}}{\pred{link}{\var{X}, \var{Y}}}\\
        \frule{\pred{reachable}{\var{Y}, \var{X}}}{\pred{reachable}{\var{X}, \var{Y}}}\\
        \frule{\pred{reachable}{\var{X}, \var{Z}}}{\pred{reachable}{\var{X}, \var{Y}}, \pred{reachable}{\var{Y}, \var{Z}}}
\end{gather*}
Next, our machine has a local network and a gateway --- a common configuration.
Traffic serviceable by the local network should be sent direct, while the remainder of routable traffic should be sent to the gateway.
\begin{gather*}
        \frule{\pred{route}{\id{local}, \var{T}}}{\pred{reachable}{\id{local}, \var{T}}}\\
        \frule{\pred{route}{\id{gw}, \var{T}}}{\neg \pred{reachable}{\id{local}, \var{T}}, \pred{reachable}{\id{gw}, \var{T}}}
\end{gather*}
In the case of this set of rules, negation has a clear interpretation.
First, generate all reachability predicates.
Then, if reachable was not derivable thus far, matches against not reachable should succeed.
However, when recursive rules and negation are mixed, it can cause termination and stability rules. Consider the set of rules:
\begin{gather*}
        \frule{\pred{p}{\var{X}}}{\neg \pred{q}{\var{X}}}\\
        \frule{\pred{q}{\var{X}}}{\neg \pred{p}{\var{X}}}
\end{gather*}
There is no interpretation of this which terminates with the fixpoint semantics, or which provides a minimal model (not discussed here).
To deal with this, Datalog imposes a requirement on negation known as stratification.
This requirement can be phrased as a graph property:

Create a graph where the predicates are nodes, and there is an edge between two nodes iff there is a rule in the program which uses the source node predicate as a premise to derive the destination node predicate.
Additionally, label all those edges where the premise predicate was negated.
There must be no cycles which include one or more negated edges.

% * Prolog
\subsection{Prolog}
While \sysname\ is primarily modeled after Datalog, some ideas more related to those in Prolog are also necessary.
%   * Forwards vs backwards chaining
\subsubsection{Backwards Chaining}
\label{sec:backchain}
The evaluation model discussed thus far is what is commonly known as forward chaining, or bottom-up search.
This approach searches for a proof by taking known facts and deriving anything they can hoping to reach the goal.
However, there are some applications for which this approach does not work well --- as an example, any system that would have a potentially infinite intensional database cannot run with such a semantics.

The other approach is known as backwards chaining, or top-down search, and is what is used in Prolog.
In backwards chaining, search is directed by a particular goal instead.
The logic engine examines the goal and the available rules, and by unification breaks it down to a series of potential subgoals, and recurses.
There are various methods to control the backtracking behavior of such a search, but they are not immediately relevant.

%   * Nontermination
\subsubsection{Nontermination}
While Datalog has termination as a key property, Prolog and most backwards chaining languages do not.
As a simple example, consider the related-symmetry rule we used before, constructed instead as a backwards chaining rule:
\begin{gather*}
        \brule{\pred{related}{\var{X}, \var{Y}}}{\pred{related}{\var{Y}, \var{X}}}
\end{gather*}
In Datalog, this terminated because the engine would fail to derive any new facts eventually.
In a backwards chaining search procedure, this could give rise to an alternating search tree.
Consider a query for $\pred{related}{\id{john}, \id{mary}}$.
This backwards chaining rule would search for $\pred{related}{\id{john}, \id{mary}}$, then by the rule, see that it could instead search for $\pred{related}{\id{mary}, \id{john}}$.
However, it would then trigger again, bringing us back to a search for $\pred{related}{\id{john}, \id{mary}}$, looping infinitely.
While this simple example could be avoided by checking if this subgoal had already occurred in the tree, more complex examples are possible which are unavoidable.\todo{citation?}

%   * TODO prolog example showing extra power

% * Circumscription
\subsection{Circumscription}
Circumscription is a concept in logic related to including Occam's Razor, or ``common sense'' into deriving a model from logic statements.
I discuss it here because it will be relevant to the \sysname\ treatment of reasoning from incomplete information.

As an example of why this might be necessary, consider the following:
\todo{Write a good example. I'm trying to avoid fluents or negation. Formula $a \wedge (b \vee c)$ is good, but need to pair real world meaning to it}

More examples are given in McCarthy's paper introducing the idea\todo{cite McCarthy's circumscription paper}.
An slightly simpler (if imprecise) way of thinking about circumscription is to consider it as taking a specification of truth which is irrefutable, while saying as little as possible --- you're not saying something grounded in fact, you're saying something nobody can prove wrong, while taking as few risks as possible.
%   * Original definition by McCarthy
\subsubsection{Original Definition}
Originally, circumscription was defined with respect to first order logic.
Specifically, the circumscription of a predicate $P$ in a formula $A$ is defined\todo{cite mccarthy}\ as
\[
(A [P/Q] \wedge \forall x. Q(x) \rightarrow P(x)) \rightarrow (\forall x. P(x) \rightarrow Q(x))
\]
McCarthy calls this a ``sentence schema'', and is essentially a statement in second order logic (since $Q$ is implicitly quantified here).
Essentially, this says that for a given formula $A$ we know to be true, and some predicate $P$ we have incomplete information about, if some other (usually artificially defined) predicate $Q$ could be used in $A$ everywhere $P$ was without changing its truth value, and $Q$ is true whenever we know $P$ to be true, then we can have the other half of the implication, making $Q$ a legitimate proxy for $P$.
Practically applying this usually involves declaring a $Q$ which is true in exactly those situations we know $P$ to be.
As a simple example, take the formula 
\[
        P(0) \wedge P(1) \wedge P(2)
\]
Define a predicate for circumscription $Q$ as
\[
        Q(x) \equiv (x = 0) \vee (x = 1) \vee (x = 2)
\]
essentially explicitly enumerating the cases in which $P$ must be true. 
Circumscribing $P$ in the example formula using $Q$ yields
\[
        P(x) \rightarrow (x = 0) \vee (x = 1) \vee (x = 2)
\]
This transformation gives more information about $P$ than was defined in the original formula.
Specifically, we now know that $\neg P(4)$.

%   * Negation as circumscription yields stratified negation
\subsubsection{Relation to Logic Programming}
Circumscription is directly connected to modern logic programming through the negation features present in many languages.
The database and rules can be considered as a single logical formula, the conjunction of known facts as predicates applied to concrete values, and rules encoded as implications.
Circumscribing over this formula will yield results consistent with a given logic program, yet also allow us to potentially derive more by giving us an explicit definition of a predicate.
For example, take the system
\begin{gather*}
        \pred{square}{\id{w}} \quad \pred{square}{\id{x}}\\
        \pred{silver}{\id{x}} \quad \pred{silver}{\id{z}}\\
        \brule{\pred{circle}{\var{X}}}{\neg \pred{square}{\var{X}}}\\
        \brule{\pred{copper}{\var{X}}}{\neg \pred{silver}{\var{X}}}\\
        \brule{\pred{penny}{\var{X}}}{\pred{copper}{\var{X}}, \pred{circle}{\var{X}}}\\
\end{gather*}
If we ask ``Is $\id{y}$ a penny?'', e.g.\ $\query{\pred{penny}{\id{y}}}$, the system without circumscription cannot answer yes.
There is no way to derive that $\id{y}$ is circular, or failing that, that it is simply non-square.
Circumscribing over both $\pred{square}{\cdot}$ and $\pred{silver}$ in the system allows us to derive $\neg \pred{square}{\id{y}}$ and $\neg \pred{silver}{\id{y}}$.
In this way, circumscription forms the basis for the form of negation present in logic languages.

% Binary analysis
\section{Binary Analysis}
% * Spiel about how binary analysis is different
While at its core binary analysis is fundamentally similar to program analysis, there are a number of challenges preventing the direct application of techniques from the program analysis world.
%   * Abstractions stripped out
Most of these challenges fall into the category of absent abstraction.
Function locations, variable locations and bounds, variable types, and control flow graphs are all generally taken for granted even in the analysis of ``low level'' languages such as C.

%   * CFG unknown
While all of these can cause difficulties, the inability to recover a complete control flow graph can be one of teh most difficult to deal with.
Nearly every algorithm for recovering any of the other missing abstractions depends strongly on having a pre-existing control flow graph.\todo{citations}
However, without the abstractions present in the source language, determining the potential targets of an indirect (computed) jump becomes somewhere between difficult and impossible.
Being unsound here can result in unsoundness in every analysis down the road, but frequently ``good enough'' approximations are used due to the difficulty of this problem.

%   * Extraordinarily stateful, with large state
One of the features of functional code that makes it easier to analyze is its thorough reduction of state.
Traditional procedural/imperative langauges make this slightly more difficult through the use of variables in many places, expanding the set of implicit inputs to each step greatly.
Binary code however, takes this to a whole new level of complexity.
With no bounds on variables, arrays, or inputs, it becomes quickly difficult to determine what parts of memory and registers are really the relevant state executing forwards from a particular point.
Due to the lack of scope or lifetime abstractions, once a new piece of state enters an execution, it is rare for it to leave at a later point --- much of the program image in memory is potential state.
There are things that can be done to alleviate this, some approximations, some sound.
However, the bottom line remains: the state space is so large that a number of traditional exploration techniques break down.\todo{cite examples}

% * Static
\subsection{Static Analyses}
%   * What is static?
Static analyses are those which focus on deriving information from the structure of the complete program.
While not always true, these are usually properties that talk about many or all paths.

%   * Control flow recovery
Control flow recovery is usually thought of as a static analysis.
The goal is to generate a complete graph of possible program counter transitions in the program while containing as few spurious edges as possible.
While information from a concrete execution can be useful to the final product (as it contains a series of jumps which the output graph must contain to be correct) it does not allow for the declaration of any control flow graph as complete.
The simple form of control flow recovery involves performing recursive descent disassembly from a given entry point, watching for a transition to a distinguished exit node (usually via a function return).
However, indirect jumps due to function pointers, C++ method calls, and even creative compilation of case statements can wreak havoc on this method, since the value being jumped to is not known ahead of time.

%   * Value analysis (Abstract Interpretation)
It follows then, that recovering upper bounds on the range of various pieces of state in a program would be useful.
One common way to do this is to perform abstract interpretation of the program over a domain which is expected to model the range of values well.\cite{vsa}\todo{cite wrapped intervals}

Unfortunately, this abstract interpretation assumes that the control flow graph has already been determined in order to sequence its transformations.
Additionally, it can be tripped up by loops which do not converge nicely in its particular domain.
The state of the art approach to this is to combine value analysis with control flow recovery, allowing new paths to be informed by new values, which in turn re-inform the value analysis\cite{jakstab}.

%   * Type recovery
Another important fundamentally static analysis is type recovery.
While there is no rule that says there must be types as we usually think of them in a binary program, they are usually present due to a source language or programmer volition.
Even assembly programmers and dynamic language users\todo{cite javascript speedup through type inference} usually use consistent types to some extent.

Since the annotations have all been compiled away, there are two primary sources for type information: propagation from code boundaries and program action.
Propagation\cite{howard} from code boundaries entails knowing a priori that a particular library or service in use by the program matches some signature, then following definitions of its arguments and uses of its results to annotate the program with this information.
This helps some, but largely fails on any kind of internal data structure.
Usage based analysis\cite{tie,bitr} tends to do better at this, and works by examining how the code defines and uses state and generating constraints for types based on that.

% * Dynamic
\subsection{Dynamic Analyses}
Dynamic analyses are those which focus on properties of individual executions, or statements of the form ``there exists a trace with the property\ldots''.
These analyses are frequently good at narrowing down/pinpointing problems, or increasing the accuracy of approximations.
However, they tend to do a poor job at making definitive statements about whole-program properties.

%   * Fuzzing
\subsubsection{Fuzzing}
One common modern way to find issues in code, especially non-introspectable code like binaries, is to simply perform a very large number of executions with varied inputs.
This technique is known as ``fuzzing'', and is widely used in the search for security flaws.
There are a number of axes along which this technique can differ, but the primary two are blackbox\todo{cite peach} vs whitebox\todo{cite grammar based whitebox fuzzing (pldi) and concolic testing} and generational\todo{cite peach} vs mutational\todo{cite zzuf}.

%   * Symbolic Execution + SMT
\subsubsection{Symbolic Execution}
A slightly more sophisticated (though more expensive) approach is to attempt to keep as many inputs as possible ``symbolic'' while executing.
Essentially, if an exact value is not needed in order to take a step, expressions based on inputs are used in place of concrete values as long as possible.
When a conditional branch is taken, the condition is added to a path constraint, restricting the range of inputs for which this execution is considered feasible.
When a potentially dangerous operation is about to be performed (such as jumping to or dereferencing a symbolic value), a check for the plausibility of the ``bad'' scenario can be fed to an SMT solver\footnote{\
SMT stands for Satisfiability Modulo Theory, and can be thought of as a SAT solver whose variables have a extra domains and additional operations. In this work, we primarily consider the theory of bitvectors.       
}.
The formula is created by taking the conjunction of the path constraint with whatever condition suffices to cause the badly behaved event.
If the SMT solver returns a solution, a bad input has been found\todo{cite mayhem+co}.
If instead of simply sufficing, the condition passed is complete (e.g.\ the operation is \emph{only} dangerous if the check passes), the SMT solver returning UNSAT gives safety, but for \emph{this path only}.
Since loops create an infinite number of paths, it is difficult to demonstrate any region safe this way, but it can rule out certain kinds of exploration.
