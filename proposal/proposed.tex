\chapter{Proposed Work}
My proposed work can be split into three areas.
I will implement \sysname, allowing for integration of analyses and experimentation.
I will prove basic properties about the logic behind \sysname\ to show it can be reasoned about.
I will run experiments to quantify the interactions between various analyses.

% Implement Holmes
\section{Implement Holmes}
% * Logic server/backing
First, I will implement the logic engine itself.
Work has already begun on this, and it supports forward chaining rules with external functions over RPC already.

% * Analyses
Secondly, I need to actually write or connect analyses to the system.
My current targets are:
\begin{itemize}
        \item Control Flow Graph Recovery
        \item Value Set Analysis\cite{vsa}
        \item Type Recovery\cite{bitr}
        \item Fuzzing
        \item Alias Analysis (undecided which)
\end{itemize}

\section{Verify Reasoning}
I intend to verify the ability to reason based on a \sysname\ database.
This boils down to a soundness property.
Since the precursor language is already sound, there are two main properties to prove:
\subsection{Monotonicity of Monotonic Aggregators}
Here, what it means for monotonic aggregators to actually be monotonic is:
For two programs $P$ and $P'$, if both $P$ and $P'$ terminate on a given goal and produce results $R$ and $R'$ respectively, and $P \subseteq P'$, then $R \subseteq R'$.

\subsection{Soundness of Hypothetical Circumscription}
I define the soundness of hypothetical circumscription to be:
If a program has an intermediate fact database and assumes the negation of some fact, that fact is not in the database.
If a fact is in the intermediate fact database, then there is either a derivation from the intensional database via the rules and call-cc.

\subsection{Terminating Fragment}
% * (Optional/extension) Subset which terminates
Should more time be available, I will attempt to find as large a fragment of the language which terminates to aid in writing preprocessing-style analyses.
This will likely take the form of attempting to separate slots in predicates which can contain the results of external code from those which cannot.
Disallowing recursion on the basis of those slots, plus some additional limitations, will likely yield a terminating fragment.

\section{Quantification of Analysis Interactions}
While I expect there to be gains from integrating analyses with logic language beyond ease of implementation, in cases other than CFG recovery and value analysis,\cite{jakstab} their interactions are not documented or measured.
To this end, I intend to measure for each implemented analysis its result quality and resource usage with and without integration with each other.
To guard against implementation differences, I intend to also measure against existing baseline implementations when available.
Essentially, if the dependency closure of analysis $A$ includes analyses $D_1 \ldots D_N$, I will  end up with datapoints for:
\begin{itemize}
        \item Full integration
        \item Each optional dependency on/off (all combinations)
        \item Integrated vs one-shot for each required dependency (all combinations)
\end{itemize}
assuming the amount of time required does not become prohibitive.
If it does, I will manually select potentially interesting subsets of dependencies to integrate/pre-run or enable/disable, but will be sure to include all off/one-shot and all on/integrated among them.

Within this data, I am looking to see improvements primarily in accuracy, though I would not be surprised to see them in resource usage occasionally.
Primarily though, I am interested in locating analyses whose combination improves more than the sum of its parts.
Specifically, if $A$, $B$, and $C$ are related, and $A + B + C$ performs well on $A$'s metric in cases where neither $A + B$ nor $A + C$ did, that is a potentially interesting finding.
