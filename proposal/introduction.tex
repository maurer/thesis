
\section{Overview}
The discipline of binary analysis focuses on the automated recovery or determination of various properties over compiled code.
This sort of analysis remains critical in the modern world as the majority of commercial software comes in this form -- compiled code with no symbols or source code.
This presents different challenges from typical program analysis, as many things which are normally available (control flow information, type signatures, variable boundaries) are now absent and must be recovered.
Many or these challenges have been tackled individually, sometimes using as input another piece of recovered information.
However, these systems have previously focused on recovering one piece of information, explicitly invoking a particular method of gaining the information they depend upon.

Some examples of properties which have been recovered statically in some form include types, function boundaries, variables, control flow, and value analysis.
Each of these has a dependency relationship with the others, but existing implementations hardcode the execution of the other analyses, preventing modular improvement of the state of the art.
In the case of Jakstab\cite{jakstab}, a control flow recovery system, the dependencies between their control flow recovery and their value analysis necessitate a tight loop between the two.
VSA\cite{vsa} provides the most accurate value analysis available today. However, it is also known for its long runtimes and occasional divergence.
For variable identification, DIVINE\cite{divine} is currently top of the pack. However, rather than accepting input from any value analysis, it is directly integrated only with the costly VSA technique.
BiTR\cite{bitr} recovers types, but consumes control flow information to do so.
It would be possible for BiTR to consume an incomplete control flow graph, but it does not do so because existing control flow recovery mechanism implementations assume they are run to completion, and then stop.
ByteWeight\cite{byteweight} uses heuristics to detect possible function starts.
Every analysis above consumes this information, but normally generates it solely via a much more expensive and incomplete control flow analysis.

In another branch of work, dynamic analyses tend to give more specific, but expensive and incomplete responses than the static systems discussed so far.
An execution trace provides much, such as concrete feasible values for variables, bounds for may-alias and must-alias analyses, and partial control flow graphs.
Other work attempts to recover data structures from concrete memory accesses, which could be fed as suggestions to a static inference engine.
Finally, symbolic execution techniques are well known to be extremely expensive, but are of interest for their ability to answer what would otherwise be very difficult questions\cite{mayhem}.

These examples demonstrate one of the major weaknesses of this body of work: these systems use results from each other, but each system only works with one question answered as an output, and acts as a standalone system.
Additionally, since each system has been considered in a vacuum, the cost vs accuracy refinement has not been examined.
These systems could all benefit from using each other's output to improve their results when possible, and from the ability to combine results of different methods for answering the same question, depending on how much precision is needed or what resources are available.

%Previous work hits roadblocks, this should address them
Previous work in binary analysis has run into a number of stumbling blocks related to getting the input information they require to actually make inferences. 
In Figure~\ref{fig:problemsAddressed}, I show where previous work has encountered these issues.
Here, dependency cycle means their novel component interacted with another cyclically.
Expensive optional dependencies indicates the use of an expensive analysis where a cheaper one would usually suffice.
Unused inputs is marked when the system mentioned other analyses that should give an improvement when combined, but chose not to do so, indicating a barrier to running the analysis and integrating it.

Logic languages stand as a strong candidate to resolve this conundrum. Unfortunately, existing logic languages take a fairly restrictive view of external predicates, generally assuming termination.
If we take a solely forward chaining view, as is done in Datalog, expensive and possibly nonterminating external predicates will quickly eat all available resources.
The version of backwards chaining employed by Prolog has explicit execution order, and so too falls victim to being too inflexible to allow the scheduling of available resources.
I will explore the space of executing logic programs with external predicates that may not terminate, and may be prohibitively expensive to use everywhere, with direct application to binary analysis.

\begin{figure*}
\begin{tabular}{|c||c|c|c|}
\hline
Analysis & Dependency Cycle & Expensive Optional Dependencies & Unused Inputs\\
\hline \hline
Jakstab\cite{jakstab} & \fyes & \fyes & \fno\\
Phoenix\cite{phoenix} & \fyes & \fyes & \fyes\\
TIE\cite{tie} & \fno & \fyes & \fyes \\
BiTR\cite{bitr} & \fno & \fyes & \fyes\\
\hline
\end{tabular}
\caption{Stumbling Blocks}
\label{fig:problemsAddressed}
\end{figure*}

\subsection{Proposed Work}
%This is what we're gonna do
\sysname\ addresses these issues by expressing the relationship between various analyses in the form of rules in a logic language, similar to Datalog or Prolog.
Existing external analyses will be bound to rule firing.
Dependency cycles can be dealt with via fixpoints when rules chain forwards, and a search procedure when chaining backwards.
Expensive procedures can be dealt with via backwards chaining rules tuned to be searched late in the search space.
Finally, presenting a unified data representation and allowing analyses to specify their own dependencies eases the difficulty of using information from other analyses if available.

%This is how we're gonna test it
I will test \sysname\'s effectiveness by using it to combine several existing analysis techniques together into a coherent whole, then measuring the performance of these analyses against the original version where possible.
I expect to see an increase in modularity and ease of implementation.
Additionally, I expect to improve output quality if some inputs were neglected.
Lastly, I anticipate a performance increase in analyses that were using a heavyweight analysis in places where it was not needed.

I will also examine the ability to reason about programs written in \sysname.
First, I will attempt to prove termination of a subset of the forward chaining fragment of the language.
Secondly, I will try to express subsumption properties required of combiners in order to guarantee that adding facts does not cause statements derived without circumscription to become no longer true.
\subsection{Extended Work}
The use of a logic language oriented approach to describing the relationship between analyses provides a unique opportunity to explore the implications of circumscription in incomplete logic langauges.
Circumscription is used in traditional logic languages at the point of saturation or search termination in order to provide a form of stratified negation.
\sysname\ and binary analysis each provide an attribute interesting for circumscription:
\sysname\ is incomplete, with rules which may never complete execution, making stratification an inadequate solution.
Binary analysis deals very frequently in ranges and various sorts of bounds on what things can be.
Within this context, circumscription may be able to formalize the notion of conjecturing ``true'' values for things normally only bounded by ranges.

Additionally, \sysname\ provides an opportunity to examine scheduling in dependent workloads with multiple potential paths.
One possible approach to this is a tactics script style langauge to allow the programmer to provide evaluation strategies to \sysname.
Another approach is to use a statistical method to allow \sysname\ to learn the relative costs and benefits of different paths within a program over time on larger workloads.

I intend to examine this avenue of work if time allows after the main body of work.
\section{Thesis}
The logic language implemented in \sysname\ provides a superior paradigm for expressing and executing codependent, incremental, or expensive analyses which provide results to each other.
\section{Timeline}
I plan to complete my thesis before Aug 2016.
\begin{itemize}[label={}]
\item {\bf Now-Apr 2015} Thesis Proposal
%CCS (Resubmit BiTR unless I've gotten a _long_ way)
\item {\bf May-Aug 2015} \sysname\ Basic feature complete (\S~\ref{sec:holmesBasic})
%NDSS (Use holmes to piggyback on Jonathan's inevitable resubmission of firmware systemwide paper?)
\item {\bf Sep 2015} \sysname / BAP integration
\item {\bf Oct 2015} CFG Recovery / VSA interleaving sample application
\item {\bf Nov 2015}
  \begin{itemize}
  \item BiTR (Type Recovery) Port
  \item Submission on incremental codependent analyses (Oakland target)
  \end{itemize}
\item {\bf Dec 2015 - Jan 2016} Work on termination and subsumption properties
\item {\bf Feb 2016} Add fuzzing campaigns to test nonterminating analyses
\item {\bf March-May 2016} Explore stretch goals, such as circumscription and scheduling 
\item {\bf May-Sep 2016} Thesis Writing
\item {\bf Aug 2016} Defense
\end{itemize}
