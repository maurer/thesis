\section{Proposal Overview}
\subsection{Proposed Work}
%Previous work hits roadblocks, this should address them
Previous work in binary analysis has run into a number of stumbling blocks related to getting the input information they require to actually make inferences. 
In Figure~\ref{fig:problemsAddressed}, I show where previous work has encountered these issues.
Here, dependency cycle means their novel component interacted with another cyclically.
Expensive optional dependencies indicates the use of an expensive analysis where a cheaper one would usually suffice.
Unused inputs is marked when the system mentioned other analyses that should give an improvement when combined, but chose not to do so, indicating a barrier to running the analysis and integrating it.

%This is what we're gonna do
\sysname\ addresses these issues by expressing the relationship between various analyses in the form of rules in a logic language, similar to Datalog or Prolog.
Dependency cycles can be dealt with via fixpoints when rules chain forwards, and a search procedure when chaining backwards.
Expensive procedures can be dealt with via backwards chaining rules tuned to be searched late in the search space.
Finally, presenting a unified data representation and allowing analyses to specify their own dependencies eases the difficulty of using information from other analyses if available.

%This is how we're gonna test it
I will test \sysname\'s effectiveness by using it to combine several existing analysis techniques together into a coherent whole, then measuring the performance of these analyses against the original version where possible.
I expect to see an increase in modularity and ease of implementation.
Additionally, I expect to improve output quality if some inputs were neglected.
Lastly, I anticipate a performance increase in analyses that were using a heavyweight analysis in places where it was not needed.
\todo{Add more entries to table}
\todo{Verify TIE/Phoenix mentioned something they could use}
\todo{Optional isn't the best word for the expensive deps}
\begin{figure*}
\begin{tabular}{|c||c|c|c|}
\hline
Analysis & Dependency Cycle & Expensive Optional Dependencies & Unused Inputs\\
\hline \hline
Jakstab\cite{jakstab} & \fyes & \fyes & \fno\\
Phoenix\cite{phoenix} & \fyes & \fyes & \fyes\\
TIE\cite{tie} & \fno & \fyes & \fyes \\
BiTR\cite{bitr} & \fno & \fyes & \fyes\\
\hline
\end{tabular}
\caption{Stumbling Blocks}
\label{fig:problemsAddressed}
\end{figure*}
\subsection{Extended Work}
The use of a logic language oriented approach to describing the relationship between analyses provides a unique opportunity to explore the implications of circumscription in incomplete logic langauges.
Circumscription is used in traditional logic languages at the point of saturation or search termination in order to provide a form of stratified negation.
\sysname and binary analysis each provide an attribute interesting for circumscription:
\sysname is incomplete, but unlike Prolog, some rules may never complete execution, making stratification an inadequate solution.
Binary analysis deals very frequently in ranges and various sorts of bounds on what things can be.
Within this context, circumscription may be able to formalize the notion of conjecturing ``true'' values for things normally only bounded by ranges.
I intend to examine this avenue of work if time allows after the main body of work.
\section{Thesis}
The logic language implemented in \sysname\ provides a superior paradigm for expressing and executing codependent, incremental, or expensive analyses which provide results to each other.
\section{Timeline}
I plan to complete my thesis before Aug 2016.
\begin{enumerate}
\item[Now-Apr 2015] Thesis Proposal
%CCS (Resubmit BiTR unless I've gotten a _long_ way)
\item[May-Aug 2015] \sysname\ Basic feature complete (\S~\ref{sec:holmesBasic})
%NDSS (Use holmes to piggyback on Jonathan's inevitable resubmission of firmware systemwide paper?)
\item[Sep 2015] \sysname / BAP integration
\item[Oct-Nov 2015]
  \begin{itemize}
  \item BiTR (Type Recovery) Port
  \item Submission on incremental codependent analyses (Oakland target)
  \end{itemize}
\item[Dec 2015 - Jan 2016] Integrate fuzzing/tracing based analyses
\item[Feb 2016] Submission on automatically refining results via fuzzing (USENIX target)
\item[March-May 2016]
  \begin{itemize}
  \item Explore circumscription in incomplete logical systems.
  \item Submission on formalization of heuristics via circumsription (CCS target)
  \end{itemize}
\item[May-Sep 2016] Thesis Writing
\item[Aug 2016] Defense
\end{enumerate}
