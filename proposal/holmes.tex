\section{\sysname\ Basic}
\label{sec:holmesBasic}
Initially, I will aim to implement a more basic set of features in \sysname.
This feature set is intended to be complete enough to pursue the majority of the application work, while simple enough that its completion should not be in question.
In the basic version of \sysname, I will implement a system with:
\begin{itemize}
\item Forwards and backwards chaining
\item Simple external predicates
\item Result caching/persistence
\item Combination functions
\end{itemize}
Additionally, I will formalize the semantics of the restricted system.

While the actual system will operate based on RPC calls, it is conceptually useful to think about the system in terms of a written out set of rules.
In Figure~\ref{fig:holmesGrammar}, I depict a draft of what this would consist of for \sysname\ Basic.
Additional constraints are necessary beyond what is expressed in the grammar (e.g. variable occurrence restrictions), but this gives the basic concept for the moment.
The only thing not directly expressible in this language is the external functions themselves.
External functions creation is not given a representation here because they are represented to the system by providing it with an RPC handle, and so they act as black boxes;
to provide a direct means of expressing them would be misleading about what the system could reason about.

\begin{figure}
\begin{grammar}

<prog> ::= (<decl> | <rule>)*

<decl> ::= <pred> `(' <argsig>+ `)' [`where' (`(' <arg>* `)' `=' <expr>)]

<argsig>
::=  <type>(= : <func>)
\alt <type>($\oplus$ : <func>)

<rule> ::= <head-constr> <sched-sym> <body-clause>

<sched-sym> ::= $\leftarrow$
\alt $\Leftarrow$

<head-constr> ::= (<pred> `(' <expr>+ `)')+

<expr> ::= <var>
\alt <consnt>
\alt <func> `(' <expr>* `)'

<body-clause>
::=  `('<body-clause>`)' $\wedge$ `('<body-clause>`)'
\alt <pred> `(' <arg>* `)'

<arg> ::= $\exists$ \alt <var> \alt <const>
\end{grammar}
\label{fig:holmesGrammar}
\caption{\sysname\ Grammar}
\end{figure}

Ignoring external predicates for a moment, I expect the semantics of the basic system to resemble a Prolog with user annotated precomputation and caching.
One notable deviation is the lack of an implicit evaluation order.
While in Prolog this provides the programmer the ability to write terminating code, in \sysname it precludes the opportunity to explore scheduling of proof search.
A second deviation is the presence of combiners.
I suspect, but am not sure, that combiners will only have a minor effect on the actual execution of the language.
I believe the effect to be minor because the way combination happens means that matching on an improved combined fact will look identical to matching on a not-yet present uncombined fact.

Additionally, I intend to investigate what circumstances lead to terminating evaluation of the forwards chaining fragment.
Adding backwards chaining would prevent reasoning from the finite set of initial facts, and so make for an undecidable problem.
Even with only forwards chaining, this is potentially tricky.
Given variables which range over an infinite domain, properties are required on the external predicates to guarantee termination.
Knowledge of what properties these are would assist future programmers of this system in designing preprocessing phases for their applications.
\section{\sysname\ Advanced}
Possible additional topics to explore as they are relevant to applications include:
\begin{itemize}
\item Retraction - removal of facts or even rules from a live system with minimal perturbation to support hypothetical analyses
\item Circumscription - reasoning as though knowledge of some fact is all you will ever have/can be known, despite not knowing this for certain
\item Tactical Evaluation - adaptation of a system similar to Coq's tactic scripts to suggest evaluation strategies
\item Learning Evaluation - self improvement of evaluation strategies based on past success and cost of external predicates
\item Resource Constrained Evaluation - Getting the best answer within certain time/machine constraints
I intend to hit at least some of these as I explore the usefulness of this system to binary analysis, but not necessarily all of them.
\end{itemize}
