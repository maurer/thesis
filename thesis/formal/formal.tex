\newcommand{\hu}{\mathbb{U}}
\newcommand{\hb}{\mathbb{B}}
\newcommand{\cons}[1]{C(#1)}
\newcommand{\decides}{\triangleleft}
% NFI = no force inconsistent
%\newcommand{\nfi}[2]{{#1} \triangleleft^? {#2}}
\newcommand{\world}[1]{W({#1})}
\newcommand{\stepass}{\leadsto}
\newcommand{\circset}{K}
\newcommand{\aggset}{A}
\newcommand{\decirc}[2]{D(#1, #2)}
\newcommand{\viol}[2]{V(#1, #2)}
\newcommand{\trule}[1]{\textsl{#1}}

\chapter{Formal Holmes}
\label{chap:formal}
Holmes differs from base datalog in callbacks (analogous to external predicates), aggregates, circumscription, and call/cc.
External predicates already have a well understood semantics, and we will model callbacks here by extension of our Herbrand universe during instantiation.
Aggregates in the presence of callbacks can be replaced by a syntactic tranformation, other than for purposes of circumscription.
Circumscription by itself yields stable set semantics when the program is consistent under it.\todo{prove or argue?}
Adding call/cc forces a solution to exist when it otherwise would not, adding a restricted set of solutions which violate the minimality constraint of stable set semantics when no stable set exists.
%TODO add stable-set (and well-founded?) to background section

I will first go over the negation model embedded at base datalog to demonstrate how it works more clearly (\S~\ref{formal:sec:negation}).
Then, I will show how to construct the Herbrand universe for a program with callbacks, aggregates, and circumscription (\S~\ref{formal:sec:herbrand}).
Finally, I will give semantics for ther Herbrand instantiation of a Holmes program which interprets circumscription as infinite negation, and call/cc as the failure of one of those negated terms (\S~\ref{formal:sec:semantics}).
%TODO add any proofs to intro
\section{Negation}
\label{formal:sec:negation}
Traditional datalog consists only of a head clause, implied by a series of body clauses.
To this, we allow body clauses to be optionally negated (head clauses must still be positive).

As we have no function symbols or callbacks, we define a Herbrand universe $\hu$ which is simply the set of constants in the program.
$\hb$ is the set of every predicate instantiated at every combination of values in $\hu$.

We instantiate the program into the Herbrand universe, making a version of each rule with every variable instantiated at every member of $\hu$.
This eliminates pattern matching and reduces all rules ot a set of rules of the form
\[
	p \leftarrow \bigwedge q_i \wedge \bigwedge \neg r_j
\]
where $p, q_i, r_i \in \hb$.
While there will be no rules of this form in the initial translation of the program, we also allow rules of the form
\[
	\neg p \leftarrow
\]
which simply asserts $\neg p$.
I will use $N$ to describe a possibly negated term, e.g. it has form $p$ or $\neg p$ where $p \in \hb$.

For a program $\Pi$ which consists of a set of such rules,
\[
	\infer[\textrm{m.p.}]{\Pi \models N_0}{N_0 \leftarrow \bigwedge_{i \in I} N_i \in \Pi & \forall i \in I. \Pi \models N_i}
\]

We will define an interpretation of negation in such programs via a Kripke frame.
\footnote{Our use of $\neg$ is considered to be beneath the formula level.
The traditional $w \not \models P$ giving $w \models \neg P$ is talking about a different operator and kind of negation than what we are examining.
We do not use formula negation here, so we do not define a new symbol for it.}

We say a program $\Pi$ is consistent if it does not model both truth and falsehood for a single fact.
\[
	\infer[\textrm{consistent}]{\cons{\Pi}}{\forall p \in \hb(\Pi). \Pi \models p \imp \Pi \not \models \neg p}
\]
We say a program $\Pi$ decides a fact if it models it as either true or false.
\[
	\infer[\textrm{decide-true}]{\Pi \decides p}{\Pi \models p}
	\quad
	\infer[\textrm{decide-false}]{\Pi \decides p}{\Pi \models \neg p}
\]
We say a program $\Pi$ does not force the inconsistency of a fact if either it decides that fact, or adding one of its truth or negation results in a world.
\[
	\infer[\textrm{nfi-decided}]{\nfi{\Pi}{p}}{\Pi \decides p}
	\quad
	\infer[\textrm{nfi-true}]{\nfi{\Pi}{p}}{\world{\Pi \cup (p \leftarrow)}}
	\quad
	\infer[\textrm{nfi-false}]{\nfi{\Pi}{p}}{\world{\Pi \cup (\neg p \leftarrow)}}
\]
Finally, a program is a world in our Kripke frame if it does not force the inconsistency of any fact, nor is it already inconsistent.
\[
	\infer[\textrm{world}]{\world{\Pi}}{\cons{\Pi} & \forall p \in \hb(\Pi). \nfi{\Pi}{p}}
\]

We define a relation between worlds $\stepass$ describing assumptions that may be made from a given program.
\[
	\infer[\textrm{assume-false}]{\Pi \stepass \Pi \cup (\neg p \leftarrow)}{p \in \hb(\Pi) & \Pi \not \decides p & \world{\Pi \cup (\neg p \leftarrow)}}
\]
\[
	\infer[\textrm{assume-true}]{\Pi \stepass \Pi \cup (p \leftarrow)}{p \in \hb(\Pi) & \Pi \not \decides p & \world{\Pi \cup (p \leftarrow)} & \neg \world{\Pi \cup (\neg p \leftarrow)}}
\]

We complete the Kripke frame by defining the accessibility as transitive closure over $\stepass$.
\[
	\infer[\textrm{refl}]{\Pi \leq \Pi}{}
	\quad
	\infer[\textrm{assume}]{\Pi_0 \leq \Pi_2}{\Pi_0 \leq \Pi_1 & \Pi_1 \stepass \Pi_2}
\]

Our Kripke is reflexive and transitive, placing it in S4 (preorder).
For ``well behaved'' programs, as described in \S~\ref{holmes:sec:circ}, we also follow G (convergence) for the subset of worlds accessible from the program, placing us in S4.2, directed preorder.

We consider a formula $F$ true for an input program $\Pi$ if $\Pi \models \dia \boks \dia F$ under this frame.

assume-false represents normal circumscription.
This rule alone will result in a frame which represents stable set negation.\todo{Prove and reference proof?}
Adding the assume-true is what differentiates our negation model.
This rule only allows the assumption of truth if assuming false would either make the world inconsistent immediately, or force an inconsistent choice for some fact down the line.
By doing adding assume-true, we gaurantee the existence of a solution to negation.\todo{prove and reference}
By restricting it to those cases where assuming falsehood would result in a forced inconsistency in any complete model, we still rule out trivial solutions like setting everything true, but in a loser way than is present in the stable-set semantics minimality over a reduced program.
This can be viewed as a locally-minimal stable set, rather than a fully minimal one.

Well-founded negation corresponds to those formulae which are modeled under the convergent subset of the Kripke frame (e.g. if $\leq$ were artificially modified to remove all nonconvergence by iteratively shrinking the relation).\todo{prove? reference}
In the case where the program is ``well-behaved'' and we have G directly, the well-founded negation result, the stable-set result, and our result are all the same. 

\todo{example?}

In Holmes, circumscription corresponds to a potentially-infinite variety of assume-false in which every version of a specific aggregate greater than the proposed circumscription is negated.
call/cc corresponds to assume-true, but is a bit more complex, since it needs to discern at least one specific assumption from the infinite negation arm which is untenable.

\section{Herbrandization}
\label{formal:sec:herbrand}
First, we define the Herbrand universe and base.
Then, we describe how to instantiate a Holmes program at this universe.

\subsection{Herbrand universe}
Instead of function symbols which produce new values, as in a traditional construction, we have callbacks and and lattice joins.
This is different from the normal construction because callbacks can produce an infinite set of new values, and both callbacks and lattice meets may produce values which are equal to existing values.
To address the equality issue, we assume that our construction is supplied with implementations of the callbacks and join operations as the real program would, and actually execute them on input values rather than creating symbolic expansions.

Define $U_0$ to be those constants present in the program, combined with varieties tupled up to the maximum arity of the provided functions.
Consider joins as equivalent to callbacks which just happen to return only one argument.
Let $F$ be the set of functions, modified to take tuples for multiple arguments, and to return their results ``flattened'', e.g. if a callback would return $a = 1, b = 2$ and $a = 3, b = 4$, its representative in $F$ returns $\{1, 2, 3, 4\}$.
Since both meets and callbacks are typed, if an input would be outside their domain, they return $\emptyset$.

\todo{I think there is something wrong with the infinities in this construction, but I want a draft.}
Given $U_i$, construct $U_{i + 1}$ by
\[
	U_{i + 1} = \bigcup_{x \in U_i, f \in F} f(x)
\]
We then combine these stages:
\[
	\hu = \lim_{j \rightarrow \infty} \bigcup_{0 \leq i < j} U_i
\]

The Herbrand base $\hb$ is constructed in the usual way, instantiating each predicate at every value of $\hu$.

\subsection{Program Instantiation}
For predicates which have aggregation, rewrite them as callback rules.
If we have $P(\tau_0 \cdots \tau_m, \sigma_0\wedge j_0 \cdots \sigma_n \wedge j_n)$ then we create a new callback
\[
	j(a_0 \cdots a_n, b_0 \cdots b_n) = \{c_0 = j_0(a_0, b_0) \cdots c_n = j_n(a_n, b_n)\}
\]
and add the rule
\[
	P(x_0 \cdots x_m, c_0 \cdots c_n) \leftarrow P(x_0 \cdots x_m, a_0 \cdots a_n), P(x_0 \cdots x_m, b_0 \cdots b_n) + j
\]

For each aggregated predicate, introduce a circumscripted version of the predicate, $P_c$.
Perform simple replacement of all circumscripted matches to $P$ with matches against $P_c$.

Now, the program looks like normal datalog but with callbacks.
For every element of the Herbrand universe, instantiate the rule, run the function concretely on the bound variables, and instantiate the head term.
This will result in a rules of the form
\[
	p \leftarrow \bigwedge q_i
\]
where $p, q_i \in \hb$.

\section{Semantics}
\label{formal:sec:semantics}
\todo{Maybe some (all?) of these sets should be in instantiation - probably $\viol{\Pi}{c}$}
We begin by defining a few extra sets relative to the initial program which will be needed for interpretation.
Let $\circset(\Pi) \subseteq \hb(\Pi)$ be the set of circumscripted facts added, e.g. they were of the form $P_c(\cdot)$.

Let $\aggset(\Pi)$ be a set identifying the tuple of an aggregated predicate and all of its non-aggregated inputs.

Let $\circset_a(\Pi)$ where $a \in \aggset(\Pi)$ be the set of circumscriptions which correspond to that aggregation specifically.
This is a partitioning of $\aggset(\Pi)$.

Let $\decirc{\Pi}{c}$ where $c \in \circset(\Pi)$ be the non-circumscribed version of the fact, e.g. if $c$ corresponds to $P_c(x_0 \cdots x_n)$, then $\decirc{\Pi}{c} \in \hb$ corresponds to $P(x_0 \cdots x_n)$.

\todo{Should I show how to construct this?}
Let $\viol{\Pi}{c} \subseteq \hb$ be those predicates which are \emph{directly} negated by the circumscription.
This contains all of those values belonging to the same aggregate which are not less than or equal to the circumscription.


Much of this is similar to the simple negation semantics, but I restate it here for clarity.

\[
	\infer[\textrm{consistent}]{\cons{\Pi}}{\forall c \in \circset(\Pi) | \Pi \models c. (\forall p \in \viol{\Pi}{c}. \Pi \not \models p) \wedge \Pi \models \decirc{\Pi}{c}}
\]
Like previously, consistency here requires that a circumscription not have any of the negations it represents violated.
Additionally, the circumscription must be supported: the aggregate must have reached the point on the lattice that the circumscription asserts will be the exact value.

Rather than deciding individual facts as we did previously, we now decide aggregates.
\[
	\infer[\textrm{bounded}]{\Pi \decides a}{c \in \circset_a(\Pi) & \Pi \models c}
	\quad
	\infer[\textrm{no-base}]{\Pi \decides a}{\forall c \in \circset_a(\Pi). \Pi \not \models \decirc{\Pi}{c}}
\]
The bounded case indicates that we have circumscribed this aggregate, and so have decided it.
The no-base case indicates that this aggregate is totally unpopulated by the program.
Since our lattices are not gauranteed bounded, no circumscription can be made, and we are considered to decide the aggregate.

This time, since our $\stepass$ is more complex, our world predicate and $\stepass$ are defined mutually recursively.
We also index $\stepass$ with the aggregate it is making assumptions in.

\[
	\infer[\textrm{world}]{\forall a \in \aggset(\Pi). \Pi \decides a \vee \exists \Pi'. \Pi \stepass_a \Pi'}{\world{\Pi}}
\]
A program is a world if for every aggregate, it either decides it, or we can make an assumption to progress that aggregate.

\[
	\infer[\textrm{circ}]{\Pi \stepass_a \Pi \cup c}{\Pi \not \decides a & c \in \circset_a(\Pi) & W(\Pi \cup c)}
\]
If $a$ isn't decided yet, and circumscribing it closed would not cause any inconsistency, we may do it.

For call/cc, let $\Pi_c$ be $\Pi$, but modified so that $\viol{\Pi}{c} = \emptyset$ and $c$ is unioned in.
\todo{Perhaps split up into separate judgements/predicates to make this rule not so huge}
\[
	\infer[\textrm{call/cc}]{\Pi \stepass_a \Pi \cup p}{\Pi \not \decides a & \world{\Pi_c} & \not \exists c'. \Pi \stepass_a \Pi' \cup c' & \not \exists p' \in \viol{\Pi}{c}. \Pi \models p' & p \in \viol{\Pi}{c} & \Pi_c \models \dia \boks \dia p}
\]
This rule says that if we cannot circumscribe on the aggregate, but if we did, the inconsistency we would arrive at would deal with this circumscription, then we can instead extend the aggregate to avoid the inconsistency.
We define $\leq$ based on $\stepass$ exactly as before.

Finally, we hold a formula to be true for a Holmes program if the program $\Pi$ it is translated into

\section{Potential proofs?}
\subsection{Finite form}
\begin{itemize}
	\item Equivalent to stable set without assume-true
	\item Equals stable set if stable set present
	\item Translated program is always a world
\end{itemize}
\subsection{Infinite form}
\begin{itemize}
	\item Translated program is always a world
	\item Connect more thoroughly to finite mode, by describing $c$ as infinite negations in a row, and commuting them, then using finite semantics
	\item Connect to depth-first search strategy implemented by real engine for
		\begin{itemize}
			\item Soundness
			\item Progress (don't repeat worlds)
			\item Termination when $G$ is present and extent of the unique answer is finite? This might still not be true.
		\end{itemize}
\end{itemize}
