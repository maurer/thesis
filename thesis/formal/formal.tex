\newcommand{\hu}{\mathbb{U}}
\newcommand{\hb}{\mathbb{B}}
\newcommand{\cons}[1]{C(#1)}
\newcommand{\decides}{\triangleleft}
% NFI = no force inconsistent
%\newcommand{\nfi}[2]{{#1} \triangleleft^? {#2}}
\newcommand{\world}[1]{W({#1})}
\newcommand{\stepass}{\leadsto}
\newcommand{\circset}{K}
\newcommand{\aggset}{A}
\newcommand{\decirc}[2]{D(#1, #2)}
\newcommand{\viol}[2]{V(#1, #2)}
\newcommand{\trule}[1]{\textsl{#1}}

\chapter{Holmes Specification}
\label{chap:formal}
Holmes differs from base Datalog in callbacks (analogous to external predicates), aggregates, circumscription, and call/cc.
External predicates already have a well understood semantics, and we will model callbacks as interpreted functions by extension of our Herbrand universe during instantiation (\S \ref{formal:sec:callbacks}).
We can represent aggregates in the presence of interpreted functions by a syntactic transformation, other than for purposes of circumscription.
Adding call/cc forces a solution to exist when it otherwise would not, adding a restricted set of solutions which violate the minimality constraint of stable set semantics when no stable set exists.

We will first go over the negation model embedded at base Datalog to demonstrate how it works (\S~\ref{formal:sec:negation}).
Then, we will show how to construct the Herbrand universe for a program with callbacks, aggregates, and circumscription (\S~\ref{formal:sec:herbrand}).
Finally, we will give semantics for the Herbrand instantiation of a Holmes program which interprets circumscription as infinite negation, and call/cc as the failure of one of those negated terms (\S~\ref{formal:sec:semantics}).
%TODO add any proofs to intro
\section{Negation}
\label{formal:sec:negation}
\subsection{Other forms of Negation}
\label{formal:sec:otherneg}
The first primary distinction to draw in negation is \emph{strong negation} vs \emph{negation-as-failure}, or NaF.
Strong negation describes having knowledge that something is not the case, whereas NaF describes knowing that we will never have proof that something is the case.
This difference is more formally defined by Gelfond and Lifshitz~\cite{strongneg}.\footnote{
This aricle refers to strong negation as classical negation.
}
In this work we will focus on negation-as-failure.

One way of interpreting negation is \emph{stable-set} semantics~\cite{stablemodel}.
In stable-set semantics, to check whether a proposed model is a stable set, we first take the reduct of the logic program under the proposed model.
To do this, we remove all negated atoms in the body which are not present in the proposed model (and so are satisfied), and then remove all rules which still have negated atoms in them (which were not satisfied).
If the model is the minimal model of this reduct (this new reduced program), then it is a stable set, or stable model of the program.
This interpretation admits an arbitrary number of potential stable models, including none.

Another approach is \emph{well-founded negation}~\cite{wellfounded}.
This approach constructs what it calls \emph{unfounded sets}.
These are sets of values which, relative to a given partial interpretation, are unprovable.
It then augments the partial interpretation with the negation of the unfounded set, and constructs a new minimal model, treating negated atoms as only true if explicitly present in the new partial interpretation.
This procedure runs until it reaches a fixed point, and produces exactly one well-founded partial model, which will be a subset of all stable models~\cite{wellfounded}.

In our case, stable-set semantics falls short due to the potential for no model, and the well-founded approach may return partial solutions if there is no stable set or multiple stable sets.
\subsection{Negation in Holmes}
Traditional Datalog consists only of a head which is implied by a series of atoms making up a body.
In addition to this, we allow the atoms of the body to be optionally negated.
In the original input program, the head may not be negated.

As we have no function symbols or interpreted functions, we define a Herbrand universe $\hu$ which is simply the set of constants in the program.
$\hb$ is the set of every predicate instantiated at every combination of values in $\hu$.

We instantiate the program into the Herbrand universe, making a version of each rule with every variable instantiated at every member of $\hu$.
This eliminates pattern matching and reduces all rules to a set of rules of the form
\[
	p \leftarrow \bigwedge q_i \wedge \bigwedge \neg r_j
\]
where $p, q_i, r_i \in \hb$.

For example, if we begin with the program
\begin{align*}
A(x) &\leftarrow B(x) \wedge C(0)\\
B(1) &\leftarrow \neg C(1)\\
B(1) &\leftarrow \neg A(0) \wedge \neg A(1)
\end{align*}

We only see the symbols 0 and 1 present.
This means $\hu = \{0, 1\}$ for this program, and $\hb = \{A(0), A(1), B(0), B(1), C(0), C(1)\}$.
Instantiating all variables in the program at all possible values from $\hu$, we get a variable-free version:
\begin{align*}
A(0) &\leftarrow B(0) \wedge C(0)\\
A(1) &\leftarrow B(1) \wedge C(0)\\
B(1) &\leftarrow \neg C(1)\\
B(1) &\leftarrow \neg A(0) \wedge \neg A(1)\\
\end{align*}
We can now remove parameters from predicates by assigning a name to each member of $\hb$.
\begin{align*}
p &\leftarrow q \wedge r\\
s &\leftarrow t \wedge r\\
t &\leftarrow \neg u\\
t &\leftarrow \neg p \wedge \neg s\\
\end{align*}
Now the program is in the general form described above.

In derived forms, we also allow rules of the form
\[
	\neg p \leftarrow
\]
which simply asserts $\neg p$.
We do not allow rules of this form in the initial program to ensure that the initial program is not directly contradictory.
We will use $N$ to describe a possibly negated term, e.g. it has form $p$ or $\neg p$ where $p \in \hb$.

For a program $\Pi$ which consists of a set of such rules,
\[
	\infer[\textrm{m.p.}]{\Pi \models N_0}{N_0 \leftarrow \bigwedge_{i \in I} N_i \in \Pi & \forall i \in I. \Pi \models N_i}
\]

We will define an interpretation of negation in such programs via a Kripke frame.
We say a program $\Pi$ is consistent if it does not model both truth and falsehood for a single fact.
\[
	\infer[\textrm{consistent}]{\cons{\Pi}}{\forall p \in \hb(\Pi). \Pi \models p \imp \Pi \not \models \neg p}
\]
We say a program $\Pi$ decides a fact if it models it as either true or false.
\[
	\infer[\textrm{decide-true}]{\Pi \decides p}{\Pi \models p}
	\quad
	\infer[\textrm{decide-false}]{\Pi \decides p}{\Pi \models \neg p}
\]

Now, we are equipped to describe our Kripke frame.
A program which decides all predicates and is not inconsistent is a world.
\[
	\infer[\textrm{world-base}]{\world{\Pi}}{\cons{\Pi} & \forall p \in \hb(\Pi). \Pi \decides p}
\]
Any program which is a subset of a world is a world.
\[
	\infer[\textrm{world-subs}]{\world{\Pi}}{\world{\Pi'} & \Pi \subseteq \Pi'}
\]
We define a relation between worlds $\stepass$ describing adding an assumption to the program on the left to arrive at the program on the right.
\[
	\infer[\textrm{assume-false}]{\Pi \stepass \Pi \cup (\neg p \leftarrow)}{p \in \hb(\Pi) & \Pi \not \decides p & \world{\Pi \cup (\neg p \leftarrow)}}
\]
\[
	\infer[\textrm{assume-true}]{\Pi \stepass \Pi \cup (p \leftarrow)}{p \in \hb(\Pi) & \Pi \not \decides p & \world{\Pi \cup (p \leftarrow)} & \neg \world{\Pi \cup (\neg p \leftarrow)}}
\]
We complete the Kripke frame by defining the accessibility as transitive closure over $\stepass$.
\[
	\infer[\textrm{refl}]{\Pi \leq \Pi}{}
	\quad
	\infer[\textrm{assume}]{\Pi_0 \leq \Pi_2}{\Pi_0 \leq \Pi_1 & \Pi_1 \stepass \Pi_2}
\]

We consider a formula $F$ true for an input program $\Pi$ if $\Pi \models \dia  F$ under this frame.
Here, $\dia$ means possibly, so this condition says that it is possibly the case that this formula holds.
We can visualize this on a directed graph with worlds as nodes and the single-step form of our accessibility relation ($\stepass$) as an edge.
This condition means that if we start from $\Pi$, our input program, we can find a connected program $\Pi'$ so that $\Pi' \models F$.
Conceptually, this means that there exists an allowed choice of assumptions so that $F$ holds.

\subsubsection{Example}
Consider the program we Herbrandized previously, extended with the initial fact $\{r\}$, calling the initial program $\Pi_0$.
Initially, we have that $\Pi_0 \models r$.
We cannot derive any more from $\Pi_0$ because $r$ is not sufficient to invoke any rule through m.p.
We proceed by considering a candidate world $\Pi_1 = \Pi_0 \cup \{\neg u \leftarrow\}$.
Through iterative application of m.p. we get
\[
\Pi_1 \models r \wedge \neg u \wedge t \wedge s
\]
$p$ and $q$ are still undecided, so we create candidate worlds
\[\Pi_2 = \Pi_1 \cup \{\neg p \leftarrow\}\]
\[\Pi_3 = \Pi_2 \cup \{\neg q \leftarrow\}\]
At $\Pi_3$, we have through iterative application of m.p.
\[
\Pi_3 \models \neg p \wedge \neg q \wedge r \wedge s \wedge t \wedge \neg u
\]
As a result, for all each predicate, $\Pi_3$ decides it, e.g. $\Pi_3 \decides p$, $\Pi_3 \decides q$, etc. either through decides-true or decides-false depending on its negation in the earlier formula.
Specifically,
\[
\forall p \in \hb(\Pi_3). \Pi_3 \decides p
\]
Since $\Pi_3$ only derives the truth of $r, s, t$, and does not derive any of their negation, by the \trule{consistent} rule we have $\cons{\Pi_3}$.
With these combined, we can apply the \trule{world-base} rule and get $\world{\Pi_3}$.
By \trule{world-subset}, $\Pi_0, \Pi_1, \Pi_2$ are all worlds as they are subsets of $\Pi_3$.

We focus on the accessibility relation next.
If we apply \trule{assume-false} to $\Pi_3$ and the predicate q, we get $\Pi_2 \stepass \Pi_3$.
Repeatedly applying it yields
\[
	\Pi_0 \stepass \Pi_1 \stepass \Pi_2 \stepass \Pi_3
\]
Starting with \trule{refl} on $\Pi_0$, we get $\Pi_0 \leq \Pi_0$.
Iteratively applying \trule{assume}, we get that $\Pi_0 \leq \Pi_3$.

Now we combine this information to show that what $\Pi_3$ models is a legal output for the input program.
Recall that
\[
	\Pi_3 \models \neg p \wedge \neg q \wedge r \wedge s \wedge t \wedge \neg u
\]
Since $\Pi_0 \leq \Pi_3$, we can additionally say that the above formula is possible at $\Pi_0$.
\[
	\Pi_0 \models \dia (\neg p \wedge \neg q \wedge r \wedge s \wedge t \wedge \neg u)
\]

\subsubsection{call/cc}
Adding the \trule{assume-true} rule is what differentiates our negation model from stable-set semantics.
This rule allows the addition of positive assumptions (i.e. assuming $p$ as opposed to only being able to assume $\neg p$).
We specifically restrict this rule so that it requires that $p$ is not decided, and $\neg p$ would prevent a complete, consistent truth assignment to predicates.
By restricting \trule{assume-true} we still rule out trivial solutions like setting everything true, but in a looser way than the minimal model over the reduct (\S~\ref{formal:sec:otherneg})
It is looser because it allows for models which are not minimal over the reduct.
When applying \trule{assume-true}, it adds a true proposition which has no derivation from the rules.
This proposition would not be present in any minimal model.
As \trule{assume-true} is only allowed in cases where \trule{assume-false} would lead to no solution, it still prohibits the trivial solution (all predicates are in the model).

\subsubsection{How Negations Differ}
We consider the program consisting only of the rule $P \leftarrow \neg P$ to concretely examine the difference between these negation strategies.

In the case of a stable-set semantics, this can have no model.
The only two candidate models are $\{P\}$ or $\emptyset$.
Under the first model, the rule is not satisfied, so the reduct is the empty program.
This reduct does not support $P$, and so it is not a minimal model, and not a stable set.
Under the second model, the reduct is $P \leftarrow$, and so it is not a model of the reduct because $P$ must hold.

%TODO add to background precise description of well-founded negation
Under well-founded negation, the unfounded set is empty, because $P$ still appears on the left hand side of the rule.
As a result, evaluation terminates immediately.
Well-founded negation returns the \emph{partial model} $\emptyset$.

Following our negation system, we begin with
\[\Pi = \{P \leftarrow \neg P\}\]
Now, we define two candidate worlds,
\[\Pi^- = \Pi \cup \{\neg P \leftarrow\}\]
\[\Pi^+ = \Pi \cup \{P \leftarrow\}\]
In the case of $\Pi^-$, we have both $\Pi^- \models P$ and $\Pi^- \models \neg P$, so $\neg \cons{\Pi^-}$.
As a result, $\neg W(\Pi^-)$.
For $\Pi^+$, we have only that $\Pi^+ \models P$, so $\cons{\Pi}$.
Since $P$ is the totality of $\hb{\Pi^+}$, $W(\Pi^+)$ by \trule{world-base}.
By \trule{world-subs}, $W(\Pi)$.
We can apply \trule{assume-true}, using $W(\Pi^+)$, $\neg W(\Pi^-)$, and that $\Pi$ does not decide $P$.
This gives $\Pi \stepass \Pi^+$.
Finally
\[
	\Pi^+ \models P
\]
Since $\Pi \stepass \Pi^+$, by \trule{refl} and \trule{assume} $\Pi \leq \Pi^+$, so
\[
	\Pi \models \dia P
\]

This simple example shows the differences between the approaches.
In the face of this kind of uncertainty, stable-set semantics will choose not to give an answer, well-founded semantics will not decide either way, and Holmes-style semantics will allow some predicates to be made true without justification to return a consistent complete answer. 

In Holmes, circumscription corresponds to a potentially-infinite variety of \trule{assume-false} in which the program assumes false every version of a specific aggregate greater than the proposed circumscription.
call/cc corresponds to \trule{assume-true}, but is a bit more complex.
With finite \trule{assume-false}, the \trule{assume-true} rule can assert as true the undecided predicate for which \trule{assume-false} failed.
Since circumscription assumes false for multiple predicates, call/cc must select a specific predicate amongst those which has itself given rise to a contradiction to assert as true.

\section{Example Input Program}
The subsequent sections will describe how to support a full Holmes program rather than the toy subset above.
The following Holmes program will be used as a input example to show how each translation step is performed more concretely.

\begin{lstlisting}[caption={Holmes Code}\label{lst:bigexample}] 
P(Int)
Q(Int)
R(Set(Int)^union)

inc: Q(y) <- P(x) + (y = inc(x))
promote_R: R(s) <- Q(x) + (s = promote(x))
check_threshold: P(x) <- R(s) + (x = threshold(s))
force_odd: Q(x) <- ~R(s) + (x = all_even(s))
\end{lstlisting}
where in pseudocode,
\begin{lstlisting}[caption={Procedural Pseudocode}\label{lst:bigfuncs}]
fn inc(x: Int) -> [Int] {
  return [x + 1]
}

fn promote(x: Int) -> [Set(Int)] {
  return [Set.singleton(x)]
}

fn threshold(s: Set(Int)) -> [Int] {
  if s.sum() > 7 {
    return [101]  
  } else {
    return []
  }
}

fn force_odd(s: Set(Int)) -> [Int] {
  if s.all(is_even) {
    return [s.max() + 1]
  } else {
    return []
  }
}
\end{lstlisting}

In Listing~\ref{lst:bigexample}, we have explicitly written the input and output variables for clarity.
In the actual existing Holmes implementation, these rules would lack those bindings.
The variables bound in the body and free in the head would imply the inputs and outputs:
\begin{lstlisting}[numbers=none]
force_odd: Q(x) <- ~R(s) + all_even
\end{lstlisting}

This program has three predicates - $P$, $Q$, and $R$.
$P$ and $Q$ range over integers.
$R$ ranges over sets of integers, and aggregates those sets via union.
Since $R$ has no non-aggregate parameters, every fact added to $R$ will aggregate.
The \texttt{inc} rule causes $Q$ to contain the next integer for every member of $P$.
The \texttt{promote_R} rule creates singleton sets from every member of $Q$ and puts them into $R$.
\texttt{check_threshold} adds $P(101)$ if $R$ has a set which sums to more than $7$.
Finally, \texttt{force_odd} will add an odd number to $R$'s set if it does not contain one by adding one more than the maximum of the set.
Since \texttt{force_odd} circumscribes on $R$, it should do this only if the final result would otherwise contain only even numbers.
The rule would do nothing on an intermediate state, unlike \texttt{check_threshold}.
\
\section{Herbrandization (Revisited)}
\label{formal:sec:herbrand}
In the previous setting, we did not have any form of external code, and so could not produce new symbols.
Now, we have both interpreted functions and lattice joins.
We first show how to define the Herbrand universe and base in the presence of these complications.
Then, we show how to instantiate a Holmes program at a Herbrand base constructed this way.
In the previous setting, this just eliminated variables in the rules.
Now, we also remove the callbacks and aggregates.

\subsection{Herbrand universe}
\label{formal:sec:callbacks}
Instead of function symbols which produce new values, as in a traditional construction, we have interpreted functions and lattice joins.
This is different from the normal construction because both interpreted functions and lattice joins may produce values which are equal to existing values.
To address the equality issue, we assume that our construction receives the implementations of the callbacks and join operations as the real program would, and actually execute them on input values rather than creating symbolic expansions.

Define $U_0$ to be those constants present in the program, combined with varieties tupled up to the maximum arity of the provided functions.
Consider joins as equivalent to callbacks which just happen to return only one argument.
Let $F$ be the set of functions, modified to take tuples for multiple arguments, and to return their results ``flattened'', e.g. if a callback would return $a = 1, b = 2$ and $a = 3, b = 4$, its representative in $F$ returns $\{1, 2, 3, 4\}$.
Since both joins and callbacks are typed, if an input would be outside their domain, they return $\emptyset$.

Given $U_i$, construct $U_{i + 1}$ by
\[
	U_{i + 1} = \bigcup_{x \in U_i, f \in F} f(x)
\]
We then combine these stages:
\[
	\hu = \lim_{j \rightarrow \infty} \bigcup_{0 \leq i < j} U_i
\]

The resulting universe will only be finite if the closure of all the functions and lattice joins over the original symbols is finite.
This is unlikely to be the case in practical programs: even adding an ``increment'' function is sufficient to cause an infinite universe here.

We construct the Herbrand base $\hb$ by instantiating each predicate at every value of $\hu$ and adding a $\bot$ value.
When determining what facts may be derived, this $\bot$ acts just like any other member of the base.
We will later make use of $\bot$ in order to indicate that a circumscription has been contradicted.

\paragraph{Example}
We start with the extensional database $P(1), P(3), P(5)$ and the program from Listing~\ref{lst:bigexample}.
Beginning with Herbrandization, we have
\[
	U_0 = \{1, 3, 5\}
\]
since no other constants are present in the rules.
Applying all function symbols possible to $U_0$ gives us
\[
	U_1 = \{2, 4, 6, \{1\}, \{3\}, \{5\}\}
\]
Repeating the process gives
\[
	U_2 = \{3, 5, 7, \{2\}, \{4\}, \{6\}, \{1, 3\}, \{1, 5\}, \{3, 5\}\}
\]
and so on, so that $\hu$ is the set of all integers $\geq 1$ and all sets of such integers.
To create $\hb$, we instantiate all predicates in against compatible values in $\hu$, giving
\[
	\hb = \{P(1), P(2), \cdots, Q(1), Q(2), \cdots, R(\{1\}), R(\{2\}), \cdots, R(\{1, 2\}) \cdots\}
\]

\subsection{Program Instantiation}
For predicates which have aggregation (\S~\ref{holmes:sec:agg}), rewrite them as rules with interpreted functions.
Recall that
\[
	P(\tau_0 \cdots \tau_m, \sigma_0\wedge j_0 \cdots \sigma_n \wedge j_n)
\]
means we have a predicate $P$, which when two instances of $P$ are known for which the first $m + 1$ arguments are equal, the latter $n + 1$ arguments will have their information merged by a lattice join operation indicated by $j_i$.

To translate the above form, we create a new function
\[
	j(a_0 \cdots a_n, b_0 \cdots b_n) = \{c_0 = j_0(a_0, b_0) \cdots c_n = j_n(a_n, b_n)\}
\]
and add the rule (\S~\ref{holmes:sec:callback})
\[
	P(x_0 \cdots x_m, c_0 \cdots c_n) \leftarrow P(x_0 \cdots x_m, a_0 \cdots a_n), P(x_0 \cdots x_m, b_0 \cdots b_n) + j
\]
where $+ j$ as in the informal Holmes description means to attach the function $j$ to run after the match to generate a set of assignments to variables in the head not bound by the body.

We define a partial ordering over facts in aggregate predicates based on the lattices used to combine their fields.
For a predicate $P$ defined as above, and
\begin{align}
	P_a = P(x_0 \cdots x_m, y_0 \cdots y_n)\\
	P_b = P(x_0 \cdots x_m, z_0 \cdots z_n)
\end{align}
we define the partial ordering as
\[
	P_a \leq P_b \leftrightarrow \bigwedge_{i = 0}^n y_i \leq z_i
\]
where the lattice defined by the join operator $j_i$ defines the partial order for the  aggregate field with index $i$.
If the first $m + 1$ fields do not match, then $P_a$ and $P_b$ are incomparable.

For each aggregated predicate $P$, introduce an additional predicate $P_c$ with the same field types.
$P_c$ will act as a concrete representation that $P$ at the same fields as $P_c$ would be a maximum on the ordering we defined.
Essentially, it says that if $P_c(\vec{x}, \vec{y})$, then if $P(\vec{a}, \vec{b})$ is true, $P(\vec{a}, \vec{b}) \leq P(\vec{x}, \vec{y})$, or there is no ordering between them.

Beginning with the Holmes program after translating aggregations as above, replace of all circumscripted matches to $P$ with matches against $P_c$ \emph{and} $P$ at the same indices.
Finally, for each $P_c$, add a rule
\[
	\bot \leftarrow P_c(\vec{x}, \vec{y}) \wedge P(\vec{x}, \vec{z}) + \textrm{P-less-than}
\]
where P-less-than checks whether $P_c(\vec{x}, \vec{y}) < P(\vec{x}, \vec{z})$ (defined via $\leq$ and $\neq$) and returns a singleton list of no assignments if so, and an empty list otherwise.
Here, we are using the $\bot$ value added to the Herbrand baseto indicate that this specific circumscriptive guess was contradicted.
This essentially adds a rule saying that an upper bound proposed by $P_c$ must not contradict known information about $P$.

After translating aggregation and circumscription, the program looks like normal Datalog but with functions attached to some rules.
For every element of the Herbrand universe, instantiate the rule, run the function concretely on the variables bound in the body, and instantiate the head term.
This will result in a rules of the form
\[
	p \leftarrow \bigwedge q_i
\]
where $p, q_i \in \hb$.

\paragraph{Example}
We now focus on instantiating the program given in Listing~\ref{lst:bigexample} based on the universe from the previous section.

Instantiating inc and promote_R gives
\begin{align*}
	Q(2) &\leftarrow P(1)\\
	Q(3) &\leftarrow P(2)\\
	&\vdots\\
	R(\{1\}) &\leftarrow Q(1)\\
	R(\{2\}) &\leftarrow Q(2)\\
	\vdots\\
\end{align*}
which is analogous to what we did in the finite example.

The guards and combination rules for our aggregated predicate $R$ look like
\begin{align*}
	R(\{1, 2\}) &\leftarrow R(\{1\}) \wedge R(\{2\})\\
	R(\{1, 3\}) &\leftarrow R(\{1\}) \wedge R(\{3\})\\
	\vdots\\
	\bot &\leftarrow R_c(\{1\}) \wedge R(\{2\})\\
	\bot &\leftarrow R_c(\{1\}) \wedge R(\{3\})\\
	\vdots\\
\end{align*}
The first half describes the instantiation of the ``union'' aggregator on $R$.
The second half has the guard rules which derive $\bot$ if the program exceeds a bound.

The rule check\_threshold translates to
\begin{align*}
	P(101) &\leftarrow R(\{8\})\\
	P(101) &\leftarrow R(\{9\})\\
	\vdots\\
	P(101) &\leftarrow R(\{1, 7\})\\
	\vdots\\
\end{align*}
Since assignments are not generated when the condition is not met (value sums to $> 7$), this is a rule for each set which has a sum more than 7.

Finally, force\_odd instantiates as
\begin{align*}
	Q(3) &\leftarrow R(\{2\}) \wedge R_c(\{2\})\\
	Q(5) &\leftarrow R(\{4\}) \wedge R_c(\{4\})\\
	Q(5) &\leftarrow R(\{2, 4\}) \wedge R_c(\{2, 4\})\\
	\vdots
\end{align*}
It again only generates rules if there was an output from the function, and adds an odd number to $R$ if it doesn't already have one by adding the max plus 1.

\section{Semantics}
\label{formal:sec:semantics}
We begin by defining a few extra sets relative to the initial program which will be needed for interpretation.
Let $\circset(\Pi) \subseteq \hb(\Pi)$ be the set of circumscripted facts added, e.g. they were of the form $P_c(\cdot)$.

Let $\aggset(\Pi)$ be a set of tuples of an aggregated predicate and all of its non-aggregated inputs.
For example, for $P(x_0, \cdots x_n, y_0, \cdots y_m)$ and all $y$ are aggregated, $P(x_0, \cdots x_n)$ for each possible value of $x_0$ through $x_n$ would be present in $\aggset(\Pi)$.
We will refer to members of $\aggset(\Pi)$ as \emph{aggregates}.

Let $\circset_a(\Pi)$ where $a \in \aggset(\Pi)$ be the set of circumscribed facts which correspond to the aggregation $a$.
For example, if $a = P(x_0, \cdots x_n)$, then $\circset_a(\Pi)$ would contain $P_c(x_0, \cdots x_n, y_0, \cdots y_m)$ for all possible values of $y_i$.

Let $\decirc{\Pi}{c}$ where $c \in \circset(\Pi)$ be the non-circumscribed version of the fact, e.g. if $c$ corresponds to $P_c(x_0 \cdots x_n)$, then $\decirc{\Pi}{c} \in \hb$ corresponds to $P(x_0 \cdots x_n)$.

Most of this construction should look familiar from the simple negation semantics.

\[
	\infer[\textrm{consistent}]{\cons{\Pi}}{\Pi \not \models \bot}
\]
Like previously, consistency here requires that none of the negations assumed by a circumscription also be present in the model in a non-negated form.
Since we added special guard rules when introducing the $P_c$ predicates, we can just check that $\bot$ cannot be directly derived to check that no circumscriptive assumption has yet been violated.

Rather than deciding individual facts as we did previously, we now decide aggregates.
\[
	\infer[\textrm{bounded}]{\Pi \decides a}{\Pi \models c & \Pi \models \decirc{\Pi}{c}}
	\quad
	\infer[\textrm{no-base}]{\Pi \decides a}{\forall c \in \circset_a(\Pi). \Pi \not \models \decirc{\Pi}{c}}
\]
The bounded case indicates that we have circumscribed this aggregate, and so have decided it.
The no-base case indicates that this aggregate is totally unpopulated by the program.
As we cannot circumscribe over it, the program decides the aggregate.
Specifically, the program has decided on the negation of all values which could make up the aggregate.

Our \trule{world-base} rule is similar to the finite case.
If a program decides all aggregates (rather than all predicates) and it is consistent, then it is a world.
\[
	\infer[\textrm{world-base}]{\world{\Pi}}{\forall a \in \aggset(\Pi). \Pi \decides a & C(\Pi)}
\]
As before, any subset of a world is also a world
\[
	\infer[\textrm{world-subs}]{\world{\Pi'}}{\Pi' \subseteq \Pi & \world{\Pi}}
\]

We define accessibility stepwise again.
If an aggregate $a$ isn't decided yet, and circumscribing it would not cause any inconsistency, we may do it.
\[
	\infer[\textrm{circ}]{\Pi \stepass_a \Pi \cup c}{\Pi \not \decides a & c \in \circset_a(\Pi) & W(\Pi \cup c)}
\]
The \trule{circ} rule is analogous to \trule{assume-false} because adding $c$ to $\Pi$ is essentially adding $\forall p \in \viol{\Pi}{c}. \neg p$.

To support the \texttt{call/cc} feature of Holmes (\S \ref{holmes:sec:callcc}), let $\Pi_c$ be $\Pi \cup c$ with all rules of the form $\bot \leftarrow c \wedge p$ removed.
In other words, $\Pi_c$ is $\Pi \cup c$, but will ignore inconsistencies derived from the addition of $c$.
\[\small
	\infer[\textrm{call/cc}]{\Pi \stepass_a \Pi \cup p}{\Pi \not \decides a & \world{\Pi \cup p} & \not \exists c' \in \circset_a(\Pi). \Pi \stepass_a \Pi \cup c' & \Pi_c \models \dia p}
\]
This rule says that if we cannot construct $\stepass_a$ using the \trule{circ} rule, but $\Pi_c$ would extend the aggregate $a$ with new information, then we can extend $a$ with the new information even though the proposed circumscription would not result in a world.
The \trule{call/cc} rule is analogous to the \trule{assume-true} rule.
It is only accessible when \trule{circ} (which was similar to \trule{assume-false}) is not, and allows adding a single $p \in \viol{\Pi}{c}$ as an otherwise-unsupported choice to allow deduction to proceed.

We define $\leq$ based on $\stepass$ inductively as before.
Finally, we hold a formula to be true for a Holmes program if after translating it as above to a program $\Pi$, $\Pi \models \dia F$.

\paragraph{Example}
In the previous section, we finished instantiating Listing~\ref{lst:bigexample} so that it was structured as normal datalog.
Call that base program $\Pi_0$.
We have extensionally that $\Pi_0 \models P(1) \wedge P(3) \wedge P(5)$.
Applying inc and promote\_R gives us that $\Pi_0 \models$
\begin{align*}
	Q(2), Q(4), Q(6)\\
	R(\{2\}), R(\{4\}), R(\{6\}), R(\{2, 4\}), R(\{2, 6\}), R(\{4, 6\}), R(\{2, 4, 6\})\\
\end{align*}
Adding check\_threshold gives additionally that $\Pi_0 \models$
\begin{align*}
	P(101)\\
	Q(102)\\
	R(\{102\}), R(\{2, 102\}) \cdots R(\{2, 4, 6, 102\})
\end{align*}
At $\Pi_0$, force\_odd does nothing, since $R_c$ is unpopulated.

In this case, we have only one aggregate: $R(\cdot)$.
As a result, there are only two ways forwards - circumscribe or call/cc that aggregate.
To use \trule{call/cc}, we need to know that circumscribing does not result in a world, so we start by circumscribing.
Define $\Pi_1 = \Pi_0 \cup R_c(\{2, 4, 6, 102\})$.
Since both $\Pi_1 \models R_c(\{2, 4, 6, 102\})$ and $\Pi_1 \models R(\{2, 4, 6, 102\})$, $\Pi_1 \decides R(\cdot)$ by the bounded rule.
However, with the addition of the circumscription, force\_odd now adds $\Pi_1 \models$
\begin{align*}
	Q(103)\\
	R(\{103\} \cdots R(\{2, 4, 6, 102, 103\})\\
	\bot
\end{align*}
One of the guard rules for the circumscription applied because $\{2, 4, 6, 102, 103\} > \{2, 4, 6, 102\}$ in the ordering induced by union.
$\Pi_1$ is inconsistent because $\Pi_1 \models \bot$.
As a result, $\Pi_1$ is not a world.

Circumscription failed in this case, so it is time to apply \trule{call/cc}.
We construct $\Pi_{1c}$ as in the rule - we delete all the guard rules for $R_c(\{2, 4, 6, 102\})$ from $\Pi_1$.
Since those were the only rules which could derive $\bot$, and $\Pi_1$ decided $R(\cdot)$, we can say that by \trule{world-base}, $\world{\Pi_{1c}}$.
In order to apply \trule{call/cc}, we now need to pick a conclusion of $\Pi_{1c}$ to pull out, identified as $p$ in the rule.
In this case, we will choose $Q(103)$.
We define
\[
	\Pi_2 = \Pi_0 \cup \{Q(103)\}
\]

At $\Pi_2$, we still don't decide $R(\cdot)$, so we try to circumscribe it closed again, this time with $R_c(\{2, 4, 6, 102, 103\})$.
Define
\[
	\Pi_3 = \Pi_2 \cup \{R_c(\{2, 4, 6, 102, 103\})\}
\]
This time, no new rules fire, since 103 is odd.
Since $\Pi_3$ decides $R(\cdot)$ and is consistent (no guard rules fire), $\world{\Pi_3}$.

Now that we have our final world, we need to work our way back to $\Pi_0$.
Since $\Pi_0 \subseteq \Pi_2 \subseteq \Pi_3$, and $\world{\Pi_3}$, by \trule{world-subs} we get $\world{\Pi_0}$ and $\world{\Pi_2}$.
Applying \trule{circ}, we see that since $\Pi_2$ and $\Pi_3$ are both worlds, $\Pi_2$ does not decide$R(\cdot)$, and $\Pi_3$'s change from $\Pi_2$ is to add the circumscription,
\[
	\Pi_2 \stepass_{R(\cdot)} \Pi_3
\]
To apply \trule{call/cc}, we need to demonstrate that $R(\cdot)$ cannot be circumscribed at $\Pi_0$.
$\Pi_1$ is not a world, so it cannot circumscribe $R(\cdot)$ at $\{2, 4, 6, 102\}$.
Picking any other value to circumscribe at would either not decide $R(\cdot)$ because it cannot prove $\Pi_0 \models \decirc{\Pi_0}{c}$ if bigger or incomparable, and would trigger an inconsistency rule if smaller.
As a result,
\[
	\not \exists c' \in \circset_{R(\cdot)} . \Pi_0 \stepass_a \Pi_0 \cup c'
\]
Since $\Pi_0 \not \decides R(\cdot)$, $W(\Pi_2)$, the above, and $\Pi_{1c} \models Q(103)$, we can apply \trule{call/cc} to get
\[
	\Pi_0 \stepass_{R(\cdot)} \Pi_2
\]
Since $\Pi_0 \stepass \Pi_2 \stepass \Pi_3$, $\Pi_0 \leq \Pi_3$.
We get as our final answer that $\Pi_0 \models \dia$
\begin{align*}
	P(1), P(3), P(5), P(101)\\
	Q(2), Q(4), Q(6), Q(102), Q(103)\\
	R(\{2\}), \cdots R(\{2, 4, 6, 102, 103\})\\
\end{align*}
