\begin{lstlisting}[language=C, float=t, caption={Example (C)}, label=lst:example-c]
char* g() {
	return malloc(1);
}
void f() {
	char* x = malloc(1);
	*x = 'a'; // Safe
	char* g_a = g();
	*g_a = 'a'; // Safe
	char* g_b = g();
	*g_b = 'a'; // Safe
	free(x);
	free(g_a);
	*x = 'b'; // UaF
	*g_b = 'b'; // Safe, but needs context
}
\end{lstlisting}

\subsection{Inter-procedural Flow-Sensitive Example}
In this section, we provide an example of the context- and flow-
sensitive analysis on a simple example shown in
Listing~\ref{lst:example-c}. To provide clarity on how updating alias
sets work, this example does not show data-flow merges, which is a set
union operation as previously described.

Listing \ref{lst:example-c} contains one real use-after-free bug.
We also show one location that is safe, but without context-sensitivity the analysis will raise an additional false positive.
Without context-sensitivity, both the allocations for \texttt{g\_a} and \texttt{g\_b} get merged, making the analysis believe \texttt{g\_a} aliases \texttt{g\_b} when it does not.

\lstdefinestyle{hilight-asm}{
    language={[x86masm]Assembler},
    moredelim=[is][\color{green}]{|+}{|},
    moredelim=[is][\color{blue}]{|!}{|},
    moredelim=[is][]{|~}{|},
    basicstyle=\footnotesize 
}

\lstinputlisting[style=hilight-asm,
caption={Annotated Flow-Insensitive Analysis},
label=lst:example-asm]{alias/example-annotated.asm}

We show the corresponding assembly, annotated with comments on the location of the UaF and alias sets, in Listing~\ref{lst:example-asm}.
Portions of the alias set hilighted in green are new bindings.
Portions in blue are bindings which have been destructively updated.
In the example, the variables are:
\begin{itemize}
	\item The stack slots of f (g has none): sp+24@f, sp+16@f, sp+8@f
	\item All used registers (other than the stack): RDI, RAX
	\item Allocations split on sensitivity, using the form dyn@addr\{stack\}
          \begin{itemize}
          \item Context-insensitive: dyn@5 and dyn@13
          \item Context-sensitive: dyn@5\{19\}, dyn@5\{25\}, dyn@13\{\}
          \end{itemize}
\end{itemize}

Comments show the alias sets after each step.
For example, right before the free on line 35, the alias set is:

\begin{figure}[h!]
\texttt{\{ RAX -> dyn@0; sp+24@f -> dyn@1;
    sp+16@f -> dyn@0; sp+8@f -> dyn@0;
    RDI -> dyn@0 \}}
\end{figure}

This shows that \texttt{RDI} is pointing to the memory location \texttt{dyn@1}, the allocation site for \texttt{x} in the source code.
Right after the \texttt{free} on line 35, the alias set changes to show \texttt{dyn@1} is now free.
The UAF detector would say any pointer that resolves to \texttt{dyn@1} is therefore a use-after-free bug, which happens on
line 53.

If we had been context-sensitive, the points-to relation at 59, the false positive, would instead be
\begin{figure}[h!]
\texttt{\{ RAX -> dyn@5\{25\}; sp+24@f -> dyn@13\{\};
    sp+16@f -> dyn@5\{19\}; sp+8@f -> dyn@5\{25\};
    RDI -> dyn@5\{19\}; dyn@13\{\} -> free@35;
    dyn@5\{19\} -> free@43 \}}
\end{figure}

The key difference here is that \texttt{RAX} points to \texttt{dyn@5\{25\}} rather than just \texttt{dyn@5}, so we can distinguish it from the freed allocation.
This allows context sensitivity to weed out more false positives.
