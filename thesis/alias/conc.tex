We presented \aliasname, a novel, open source, binary-only use-after-free detector.
In \aliasname, we demonstrated how to adapt alias analysis techniques from source-level analysis to compiled code.
This reduced the reliance of binary analysis on VSA in cases where its full power is not required.
We evaluated three points in the design space of alias analysis for purposes of detecting use-after-free bugs, comparing both the performance cost and precision of different sensitivity combinations, including no sensitivity, flow sensitivity, context sensitivity, and field sensitivity.
Over known-vulnerable programs, flow sensitivity removed 90\% of false positives, with context sensitivity removing an additional 93.1%.
When testing believed safe programs, we observed 93.1\% and 43.5\% reductions respectively.
Use of analyses lighter than VSA allows for whole program analysis of medium-sized programs for use-after free vulnerabilities, with additional computational resources allowing for increased precision.
