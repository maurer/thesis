%Related Work
\section{Related Work}
\label{sec:related}
%  Starting blurb similar to background blurb, surprised by this...
As this work stands at the confluence of compilation, instruction set architectures, static analysis, and type theory, there is a great deal of prior work that provided the foundation to create \bitr. There have been other attempts to perform binary type recovery. Type theorists have explored the relevant formal systems that enable us to appropriately describe the constraints imposed by the wide variety of instructions. Others have tried to build decompilers, each of which contains at least an attempt at type reconstruction.
%  Chunk related work into blobs and describe, reference to sections
\subsection{Types in Compiled Code}
In large part, previous work has considered dynamic approaches, which use execution traces to get information about concrete values. Another school of thought takes a more forensics-oriented approach, attacking the problem by looking for known data structures within a dump or trace. Finally, there is the school to which this work belongs, static type recovery, where the approach regenerates type information from a representation of the code, rather than from sample runs or matching known data structures.

\noindent {\bf Dynamic Type Description.}
Rewards~\cite{rewards} takes a dynamic, trace-oriented approach to the problem, taking execution traces and known system calls, and propagating types from system calls through the trace's reads and writes. The dynamic approach has the advantage that the analysis can know what values a memory location or register actually held at a given program point. Additionally, the dynamic approach does not have to solve the problem of indirect jumps, as when working with traces the next instruction is precisely computed. Finally, since Rewards had exact aliasing information via pointer values on each trace, flowing information from the system call barrier (their major outside source of types) is easier. While Rewards seems limited for types not present during the crossing of the system call barrier, its dynamic approach, and information from the crossing of the system call barrier could provide additional constraints to a system like \bitr\ to further improve accuracy. Howard~\cite{Slowinska2011} extends the work of Rewards by focusing on access patterns instead of simple propagation, and annotating variables from the original code, rather than locations on a dump.

\noindent {\bf Type Forensics.}
Another approach known as shape analysis~\cite{August, Haller2013a,White2013,Jung2009,Cozzie} uses dynamic traces to generate shape graphs, which they then analyze to make guesses at the types of memory locations. The systems generate the shape graph by first generating a trace, then matching the access pattern to the simplest possible graph of type structures. Some generate this trace from the compiled program, and some must annotate the program prior to compilation to achieve this trace. Once the system generates the shape graph, it compares the shape graph to multiple possibilities of what the data structure might be in attempt to classify it e.g. as a binary tree or linked list. One benefit of this technique is that when the system finds a match, more information on a name of the data structure may be available. However, if the program uses a data structure not expected by the system, some of these methods will fall short. For example, MemPick will report it to be a generic graph. It also suffers from the standard dynamic analysis issue of being unable to generate types for paths the test cases did not drive it down.

\noindent {\bf Static Type Recovery.}
Like TIE~\cite{tie}, we built \bitr\ on BAP~\cite{bap}, and also took the approach of trying to generate ranges of constraints. However, TIE performs much worse under our metric, which we feel more fully represents accuracy of more complicated types. TIE's metric is problematic for the reasons described in~\cite{sw}, but the proposed replacement metric is still dependent on a notion of distance. TIE is also slow, which hobbles its use as a large scale analysis tool. The use of DIVINE's methods was one of the bottlenecks, which we avoided in \bitr\ by recovering the type of everything that is addressable through the registers or a constant integer used along a dataflow that ended in a read or write. While the type system of TIE included structure types, TIE would rarely infer them, though its metric did not demonstrate this. Finally, if run on a static binary (e.g. without dynamic library hints), the amount TIE could infer itself was minimal.

SecondWrite~\cite{sw} instead takes the approach of lifting to a LLVM-based IR~\cite{llvm}, then using \texttt{mem2reg} to detect variables and LLVM pointer analysis to compute the types. Their reconstruction is simpler and faster than TIE's, but the approach has issues: \texttt{mem2reg} is a nice shortcut, but has the problem that \texttt{mem2reg} will not promote anything which has a use other than a load or store~\cite{llvm}. As a result, if on-stack references are in use, those stack slots will not be properly promoted to variables. Additionally, dependence on pointer analysis leaves them without a way to detect recursive types within their framework, and makes nested structures unlikely to work.

Another work focusing primarily on structure recovery~\cite{comprecon} approached the problem from the angle of figuring out what idioms compilers used to address arrays and structs, and then tried to reconstruct structs and arrays. However, by the authors' own admission, this approach cannot handle nested structs. Additionally, their dependence on assumptions about how the compiler will act and how the source language must work cause the output to be of limited use for understanding properties of code which was not necessarily built by the compilers or language expected.

\subsection{Type Theory}
Some of the inspiration for this form of type characterization \S\ref{sec:typesys} came from intersection typing~\cite{Jim1995,Shao1993}. Though we did not end up using intersection types for inferrability purposes (even the decidability of the inference turns out to be difficult and limited~\cite{interdecide}), this work informed our choice of a constraint-intersection \S\ref{sec:infmeth} approach instead of type-intersection approach.

Earlier efforts to generate typed assembly~\cite{tal,stal} also bear similarity to our work. Typed assembly language methodologies are attempting to assign types to the registers in compiled code during compilation. Some of the TAL ideas are applicable, and still others could potentially help in future reconstruction work as safer types.
However, the majority are inappropriate for the work because the compiler or author must make the code conform to the system, rather than the system describing the code.

\subsection{Decompilation}
One of the main applications for type reconstruction is decompilation. Some approaches~\cite{tydecomp} even suggest that the type reconstruction can help guide the decompilation itself rather than simply being a set of annotations applied at the end. This idea has existed~\cite{dolgova2009automatic} in decompilation for a while, but progress has been slow. More recent decompilers~\cite{phoenix} have used some of the other research~\cite{tie} in the area to improve their results as well. Given the poor state of affairs in Hex-Rays~\cite{ida}, more work in this field could improving the usability of much of the decompiler work would not be surprising.
