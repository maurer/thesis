\chapter{Holmes}
\label{chap:holmes}
% We've argued why Datalog, now argue why we want these extra features
Holmes is a dialect of Datalog, tailored with extensions for the specific use case of analyzing compiled code.
Specifically, a normal dialect of Datalog will fall short on several tools desired by the analysis author:
\begin{itemize}
		%TODO expand list
	\item Data structures
	\item Aggregation
	\item Negation
\end{itemize}

\section{Feature Selection}
To address these lacks, we add a few features to base Datalog.
%TODO once expanded list is there, do short "X address Y and Z, Q addresses..." blurb
\subsection{Callbacks}
%TODO jam widening in dataflow somewhere in here as a justification?
% Why do we need External Predicates
%TODO rewrite condition - "computation" too vague
Tasks which do not involve a fixpoint, but do involve computation, can frequently be both more difficult and more expensive to write in pure Datalog.
For example, parsing an ELF and splitting it up into segments, sections, generating records for its symbols, etc. could in principle be written in Datalog.
However, this would be difficult to write (operating on a string as a linked list, or similar structure), slow to compute due to many-way joins, and would require that the input first be transformed before even entering the program.
Other similar examples include lifting (translating a sequence of bytes into a semantics language), concrete execution, and arithmetic.

Previous approaches have noted that many of these steps come towards the beginning of analysis, and perform these tasks as a precompute phase before handing the results to Datalog to process.
In our case, we are trying specifically to avoid such phasing.
The lifter might be needed again for previously undiscovered code.
The loader might be needed again if we discover a call to \texttt{execve} and wish to follow it.
Doing a phased Datalog prevents the easy interleaving of these functionalities into the global fixpoint.

Datalog predicates are not always the best data structure for all tasks.
Datalog predicates can effectively be viewed as an append-only, deduplicated, index-by-anything table for each predicate with rows corresponding to true instances of the predicate.
This structure is very versatile, and can represent a wide variety of concepts.
However, some concepts are better represented in other ways.
One example is ILs and ASTs.
As frequently nested, branched structures, they \emph{can} be represented in Datalog, but walking one would take a very large number of lookups compared to using a traditional algebraic data type approach, not to mention the clumsiness.
Other similar concepts include formulae (as in SMT) and any kind of variable-length buffer representation.
All of these can be done in pure Datalog, assuming appropriate preprocessing has been done.
However, the resulting time and space costs make this something to be avoided.

To address the above two, we add the ability to register a callback to a Datalog rule.
If specified, whenever a that rule fires, the corresponding callback will be used supplementarily to determine the values to substitute into the head.
This allows use of traditional functional or imperative style code to implement data structure transformations or perform operations which would be slow to do in the base Datalog.
Additionally, it allows us to more readily incorporate existing code (such as the BAP~\cite{bap} lifter) rather than rewriting it from scratch.

This is equivalently powerful to external predicates in other languages in terms of expressivity.
Any callback specified could instead be turned into an external predicate and simply appended to the query.
A query involving external predicates might need to be split up into phases to be expressed in callbacks.
If an output variable of an external predicate is present in another term in the query, one would need to do a secondary join after evaluating the external predicate.
As the callback only occurs at the \emph{end} of a query, there will only be one join per query.
The callback restriction simplifies the design of the Datalog engine (the join engine is entirely separate from the callbacks), at the cost of the ability for a sufficiently advanced engine to better optimize such queries.

\subsection{Monotonic Aggregation}
Traditionally in Datalog, match clauses may only access a fixed (though arbitrary) number of facts at the same time.
Even counting can be difficult.
To verify that there are at least three instances of some predicate p, one would normally write:
\begin{verbatim}
p(x) & p(y) & p(z) & neq(x, y) & neq(y, z) & neq(x, y)
\end{verbatim}
The size of this query grows as $n^2$ in the number of elements to be counted.

This same difficulty occurs when encoding a dataflow or abstract interpretation algorithm into Datalog.
When two branches come together, a new fact representing the state with the meet lattice operation for the chosen domain applied needs to be generated.
If we do this naively, simply matching on the existence of two states at that program point and generating a new one by merging, the resulting runtime will be super-linear in the number of performed meets.

In existing systems~\cite{doop1} this is dealt with by ensuring the state in question can be extended simply by adding more facts.
This solution works in some cases, but it prevents the use of data structures like strided intervals~\cite{vsa} or widening operators in dataflow algorithms which lack finite descending chains.\footnote{
A lattice is said to have finite descending chains if at any point at the lattice, there are a finite number of values less than it on the lattice.
This property is desirable for a dataflow analysis because it guarantees termination.
When finite descending chains are not present, a dataflow analysis can provide a ``widening'' operator, which can force the termination of the system by moving farther in the lattice under heuristic conditions.
}
This is because all of these situations require reasoning about a variable sized subset of the data to make their conclusion, not just a fixed window.

Finally, external solvers often need to receive all the inputs up front, rather than incrementally.
Calling out to an SMT solver will not work if the formula from symbolic execution is stored as facts in a Datalog representation; the program would first have to walk them with a rule and a callback (or a rule and an external predicate in another system) to build up a viable representation and hand it off.
The same is true even of simpler concepts, like applying Steensgaard's algorithm~\cite{steensgaard-alias} to a set of constraints - the algorithm will either need to process all constraints at once, or it will end up store incremental program states in the database as well, ending up back at the $n^2$ problem.

Traditionally, this is dealt with by applying a post-processing step to the Datalog computation.
After rules have been executed to saturation, a query is run, and the aggregation is performed by an outside-of-Datalog program.
As stated earlier though, we want all portions of the analysis to be able to trigger all others to avoid explicit phasing.

Some of these specific scenarios can be worked around with via clever rules, but they do not apply universally.
For example, the counting check might instead use a greater than operator instead of not-equals, assuming that field is ordered.
The resulting query would then only have linear size in the count to check against.
However, this construction still only handles counting a fixed number of unique results.

To address the issue of combining information from multiple facts efficiently, we allow for predicates with aggregation.
If a predicate is declared with aggregation, a provided meet operator will be used to aggregate submissions to each aggregate field for which all non-aggregate fields match.
In the case of counting, we simply use set-union as our meet operator.
For dataflow or abstract interpretations, we can have parameters like program location be non-aggregate fields, while the state is an aggregated one.
Programs using this feature need to be aware that they may receive the aggregation of any subset of derived facts, and are only guaranteed to ever receive the aggregation at the fixpoint.
%TODO add forwardref here for why

\subsection{Hypothetical Circumscription}
Some questions revolve more around what isn't there than what is.
For instance, if \texttt{ud2}\footnote{
LLVM inserts this instruction to denote unreachable code, and is intended to cause a trap if hit
} is found in the binary, we might wish to determine if it is in fact statically unreachable.
This requires us to be able to state that we know \emph{all} of the edges entering that basic block\footnote{
A basic block is a sequence of instructions with exactly one entry point and one exit point.
Other code in the program does not jump to the middle of the block, and execution does not leave the block until the end.
}, not some subset.

As a more concrete application, if we have an algorithm which works on the SSA\footnote{
SSA, or single static assignment, is an intermediate representation frequently used by compilers to make explicit the dependency of variables on previously computed values.
Each variable in SSA form is defined precisely once in the program, and additional variables are created for cases involving mutation.
When multiple control flow paths to define a variable exist, a $\phi$ expression is used, selecting a previous definition indexed by the path taken to arrive at the current block.
} representation of a function, creating an SSA representation of that function requires the entire control flow graph.
If we add edges later, conclusions derived from the incomplete SSA form might become incorrect.

Traditional Datalog either disallows negation, or allows it through explicit stratification.
In the context of doing program analysis on binaries, we might wish to avoid this even when reasoning purely monotonically.
Consider an analysis which determines whether a function will never return.
This information is important in analysis of a calling function because it should not expect control to proceed past the called function.
To declare that a function will never return when called, we must know \emph{all} the paths within it, not just some of them.
As a result, we are implicitly talking about knowing the negation of additional edges in the control flow graph.

If we employed stratified negation, the system needs to declare an entire predicate saturated, not part of one.
As a result, to reason based on the absence of any control flow edge, the system would need to assume saturation of the entire control flow graph.
This leaves us unable to employ information that a called function will never return in the CFG definition for a calling function.

To address this need, we add hypothetical circumscription.
The core concept is that we can at need assume that a particular chunk of information is expanded, and reason forwards.
In the event that this turns out to be false, we can retract that assumption, and reason forwards again.
This allows us to deal with cases of negation which are not trivially stratified.
In the language, this feature is implemented as the an additional kind of match clause.
If a rule circumscriptively matches on an aggregated predicate, the resulting computation will be as if it matched only on the aggregation which is present in the final fixpoint database.
Circumscription is in contrast with monotonic aggregation, where rules must have correct operation for any subset of the possible aggregations along the way.
This can be used to implement the stratified case in a straightforward manner, and also to support dynamic negation as describe in the never-returning example above.

\subsection{Call/CC}
\label{sec:motive-callcc}
The astute reader may have become worried at the end of that last subsection, as it is not without reason that the standard approach to negation involves stratification.
In the case that we have a negated inference cycle not just on predicates, as stratification prohibits, but on the actual facts, the approach as described so far would lead to alternation and indeterminacy.

Interestingly, there is actually a use case for allowing negated cycles in program analysis.
In the case of the outputs of an dataflow analysis or a control flow recovery, we will need to circumscribe over their results in order to know we have received the actual output.
Using an incomplete run of an alias analysis, for example, would result in too-small upper bounds being used.
However, due to the interconnectedness of these examples, the complete alias analysis information might alter the control flow graph by refining the information used to determine the target of indirect jumps.
Changing the control flow graph would in turn invalidate the circumscribed alias analysis.

Strictly looking at the system thus far, this would loop.
Altering the control flow graph would retract the alias analysis, which would retract the alteration to the control flow graph.
Thinking about what a human would do if they had gone down the same reasoning path points to a potential solution.
If the analyst had assumed they had been shown the totality of control flow, and from that, came to find a new control flow edge, the analyst would decide that their initial assumption must have been wrong, and that edge really is there.
Essentially, this step is $(\neg P \rightarrow P) \rightarrow P$.

This matches the type signature of \texttt{call/cc}\footnote{
	\texttt{call/cc} stands for ``call with current continuation'' and is a programming technique originating in Lisp.
	It allows a function passed to it to receive as an argument a function corresponding to the rest of the program, known as the continuation.
	It is commonly used to represent failure and backtracking, but is not limited to those uses.
}, and not without reason.
In this case, the continuation is the reasoning strategy forwards, assuming P can be determined to be true.
If P is not determinable to be true, this continuation cannot actually be invoked, and we never go down that path.
If P can be determined to be true, in a traditional programming language we might go down that path.
In our case, we are constantly watching for P to be determined to be true, and if it is, we immediately take the continuation.

This feature is invisible to the user other than for performance characteristics.
The user need only specify their rules as usual, circumscribing over things which need to be complete, and if this \texttt{call/cc} condition arises, it will be automatically dealt with by reasoning forwards from the new (expanded) circumscription, after retracting relevant other derivations.

\section{Informal Semantics}
Before giving a more formal treatment of what outputs are correct for a given Holmes program, we describe how each feature functions informally, as in a programmer's guide.
Afterwards, there is a simple traced execution of a toy program which employs all of these features.
\subsection{Initial Syntax}
In Holmes, there are three kinds of possible statements, predicate declarations, rule declarations, and queries.
In a predicate declaration, a predicate is given a name, and a series of typed fields.
When done in a tuple-predicate style, this looks like:
\begin{verbatim}
derefs_var (Var, Location)
\end{verbatim}
where \texttt{Var} and \texttt{Location} are types in the embedding language.
This style can be used in cases where the meaning of each field is obvious.
In larger projects or more complicated predicates, it is preferable to name fields.
A predicate with named fields is written as
\begin{verbatim}
derefs_var {var: Var, loc: Location}
\end{verbatim}
To avoid confusion, a predicate may only be defined in one of these two styles.

Rule declarations contain a rule name, a head, and a body.
The rule name is used purely for error reporting, debugging, and provenance reporting.
It does not have any effect on the operation of the program.
A body consists of a series of atoms, joined by \texttt{\&} symbols.
The order of atoms in the body has no effect on the program.
Matches are written in one of two styles depending on the style of the predicate.

If the predicate is a tuple style predicate, we can write
\begin{verbatim}
tuple_predicate (w, ~1, _, q)
\end{verbatim}
This unifies the first field with the variable w, the second field with the constant 1, leaves the third field unconstrained, and binds the last field to the variable q.

In the field style, we can instead write
\begin{verbatim}
field_predicate {w, x: ~1, z: q}
\end{verbatim}
assuming the predicate has fields named w, x, y, and z.
This implements the same match as above.
Matching on field predicates introduces two shorthands.
First, unused fields need not be named (y in this case).
We could write \texttt{y: \_}, but it is unnecessary, making it easier to work with higher arity predicates.
Secondly, referencing a field with no binding implicitly binds it to a variable of the same name (\texttt{w} expands to \texttt{w: w} in the example).

Heads are written exactly as a body atom, except that unbound fields and fields bound to variables not defined by the right side of the rule are disallowed.
Putting it all together, a rule looks like
\begin{verbatim}
reaching_propagate: reaching {loc, var} <-
    reaching {loc: prev, var}
  & succ (prev, loc)
  & unchanged {loc: prev, var}
\end{verbatim}
Rules written like this act as normal Datalog rules, interpreted with a fixpoint semantics.
If facts are present in the database so that an assignment to variables exists for which the right side is all present, then the left hand side will be added to the database.

Queries are essentially named body, which may be evaluated as a way of querying the program state by the embedding language.
The name of a query determines the name of the function which will appear in the interface of the resulting database object.
Queries are written
\begin{verbatim}
?all_reaching: reaching {loc, var}
\end{verbatim}
When the database's \texttt{query\_all\_reaching} function is called, it will respond with a set of possible assignments to \texttt{loc} and \texttt{var}, essentially returning the current state of that predicate to the user.

Facts are inserted at the level of the embedding language, the database can be run for an arbitrary number of steps (or to fixpoint), and queries can be performed whenever the database is not stepping.

\subsection{Callbacks}
\label{holmes:sec:callback}
Callbacks are used in Holmes to allow the use of more traditional procedural or functional style code for some rules.
After writing a rule, add \texttt{+ f}, where \texttt{f} is the name of the callback to be invoked.
When the rule is considered, it will first try to match its body.
If this succeeds, the variable assignments which match will be passed, one at a time, to \texttt{f}.
\texttt{f} will return for each input assignments a list of assignments to those variables present in the head, but unbound by the body.
If there are no variables left undefined in the head, whether \texttt{f} returns a least an empty assignment structure (e.g. a list of one element, with that element being the null assignment) determines whether the rule will actually produce its head.
For example, in the rule
\begin{verbatim}
simple_func: p(y) <- q(x) & r(x) + f
\end{verbatim}
\texttt{f} would be expected to take in the value of \texttt{x}, and return a list of values for \texttt{y}.
If we wrote
\begin{verbatim}
check_even: special_even(x) <- special(x) + is_even
\end{verbatim}
A \texttt{is_even} function which checked the last bit of x, then either returned \texttt{[\{\}]} (a list containing the empty binding) for true or \texttt{[]} (an empty list) for false could implement this rule.

\subsection{Monotonic Aggregation}
\label{holmes:sec:agg}
Monotonic aggregation defines a new way to declare predicate fields.
When declaring a field, a caret (\texttt{\^{}}) followed by a function name in the embedding language with signature $(T, T) \rightarrow T$ where $T$ is the type of the field may be provided.
Such a function should be a lattice meet operator - it should be associative, commutative, and idempotent.
This function acts as an ``aggregator'' for that field.
Whenever a new fact is added which matches the non-aggregated fields of a another fact, they will be combined according to this aggregator.
Because the order of rule execution is up to the engine, this means the rule may receive multiple, incomplete aggregations along the way, or it may only receive the final aggregation - the only guarantee is that if run to a fixpoint, the rule will have received the final (largest) aggregation.

For example, say that we wrote the program:
\begin{verbatim}
p(i32, IntSet^union)
q(i32, IntSet)
promote: q(x, ints) <- p(x, ints)
?result: q(x, ints)
\end{verbatim}
and then inserted facts
\begin{verbatim}
p(0, {1, 2})
p(1, {2, 5})
p(0, {1, 3})
p(0, {2, 4})
\end{verbatim}

When calling \texttt{query\_result} after the program completed, we would be guaranteed to always see the assignments \texttt{(0, \{1, 2, 3, 4\}), (1, \{2, 5\})}.
However, we might also see present \texttt{(0, \{1, 2\}), (0, \{1, 3\}), (0, \{2, 4\}), (0, \{1, 2, 3\}), (0, \{1, 2, 4\})}.
Their presence in the output is up to the discretion of the evaluation engine.

Generally, this kind of aggregation is useful in cases where a rule wants to operate on all the information that is available, but future information will not make any of its actions incorrect.
Imagine \texttt{p} in our example as having a first field representing a variable, and the second representing values it held in a specific evaluation of a program.
In this case, the aggregation allows us to query for the set of values we know are possible to find in that variable.
Our inferences will not become wrong in the future, because the predicate describes a lower bound, which is allowed to move up (via union).
\subsection{Hypothetical Circumscription}
\label{holmes:sec:circ}
Hypothetical circumscription extends monotonic aggregation by allowing us to only examine the largest aggregation we find.
Circumscription is written on an atom by prepending a tilde to the predicate name.
If we extend the example from monotonic aggregation with a circumscriptive match on p, e.g.
\begin{verbatim}
promote_complete: r(x, ints) <- ~p(x, ints)
\end{verbatim}
\texttt{r} will contain only the pairs $(0, \{1, 2, 3, 4\})$ and $(1, \{2, 5\})$ once the program is done executing.
Examining the database in the middle of execution may yield different results, but the engine will correct these before a fixpoint is reached.

This tool is intended to be used to query an aggregation which aggregates in a different lattice direction than the one it is queried.
As a concrete example, consider abstract interpretation.
Abstract interpretation assigns bounds to a variable which get larger and larger as the computation proceeds.
This can be monotonically queried to see if something is going to be in bounds - this will only ever go from false to true as the bounds expand, never the other direction.
However, the most useful part of an abstract interpretation bound are the values it rules out - those which are outside the bounds.
In order to know that a value is outside the bounds, we need to know that we are looking at the final aggregate for that particular domain, and there circumscription becomes important.

\subsubsection{Well Behaved Circumscription}
In the vast majority of programs, circumscription will not bring any surprises to the table.
However, it is possible to encode a notion of choice using circumscription, causing the fixpoint evaluator to choose between one of two possible worlds.
For example, consider the program
\begin{verbatim}
p(IntSet^union)
q(IntSet^union)
big_p_world: p(~{2}) <- ~q(~{1})
big_q_world: q(~{2}) <- ~p(~{1})
\end{verbatim}
and insert both \texttt{p(\{1\})} and \texttt{q(\{1\})}.
The engine will now either output \texttt{p(\{1, 2\}), q(\{1\})} or \texttt{p(\{1\}), q(\{1, 2\})}, and what it does is implementation defined, and may even differ from run to run.

This nondeterminism is usually not desired, but there are currently no checks to detect it, either at compilation time or at run time.
Stratifying~\cite{prologbook} circumscription will avoid this, but part of the strength of circumscription is to allow exactly unstratified negation.
This can occur when there are two facts (not predicates) which have in their derivation a dependency cycle containing two circumscriptions.
Even if this is the case, it may not force this nondeterminism, depending on the action of the rules.
The condition where one circumscription is present is dealt with via \texttt{call/cc}, presented next.

Even if a program has has nondeterministic circumscription, all is not lost.
The engine will still emit a single, consistent world in which all constraints described in the program hold.
In many cases, this may be sufficient, and the nondeterminism embedded in the internals of the computation, rather than in the output.

\subsection{call/cc}
\label{holmes:sec:callcc}
The call/cc feature does not add any new syntax; it extends the interpretation of circumscription to deal with the case where simple fixpoint evaluation will fail to find an output set that complies with all provided rules.
Consider the program
\begin{verbatim}
p(IntSet^union)
inconsistent: p(~{2}) <- ~p(~{1})
\end{verbatim}
where we add \texttt{p(\{1\})} to the database initially.
Naively, we would oscillate between the two states \texttt{p(\{1, 2\})} and \texttt{p(\{1\})}, never finding an answer.
With call/cc, we instead interpret \texttt{p(\{1, 2\})} only to be the correct result moving forwards.
Essentially, the engine will assert \texttt{p(\{2\})} on the basis that we cannot proceed without \texttt{p(\{2\})}.
More specifically, matching \texttt{~p(~\{1\})} is matching on, among other things, $\neg p(\{2\})$.
From this, we derive $p(\{2\})$.
This fits the form of call/cc, so we add $p(\{2\})$ to the database, with call/cc as the provenance rather than a rule.

In general, this feature is intended so that a circumscription is allowed to extend itself.
For example, if an analysis assumes it will be provided the complete control flow graph (circumscription) in order to perform SSA, then determine that a new value is possible for an indirect call (extending the control flow graph), call/cc is the component that allows the engine to retain the new control flow edge despite the fact that the old SSA form needs to be retracted.

\section{Implementation}
\label{holmes:sec:impl}
In order to evaluate the language as a means towards program analysis, we need a running implementation.
\subsection{Holmes (Old Implementation)}
Initially, I produced a database backed implementation which compiled down to a combination of Rust and SQL (initially C++ and SQL) and had Postgres handle joins, deduplication, and data storage.
This had the advantage of being able to handle significantly larger working sets in theory, but in practice had significant performance issues which lead me to change approaches.
Despite this, I feel it is worth discussing here both because the failures of the implementation point out some of the unique challenges and simplifications that can be made in evaluating datalog, but also because it seems inevitable that to analyze programs substantially larger than those examined in this thesis, either a distributed platform or a disk-backed system will need to be used.
It is my hope that these lessons learned will help a future external-database based implementor avoid the same pitfalls.
Most of the details here are focused on Postgres, but other systems take a generally similar approach so similar problems are likely to occur.

As a result, this section is mostly focused on what went wrong, rather than on how the system was constructed.
If you want to see how the system was constructed, source is available at \url{https://github.com/maurer/holmes}, but be aware that it does not represent a complete implementation of the language.
In particular, it only has partial support for aggregation, and no support for circumscription.

\subsubsection{Indices}
% Which indexes to make?
%TODO cite postgres/mysql/mssql?
Database software usually does not know which indices would be ideal to keep, and since keeping extra indices is is expensive in both time and disk, most SQL systems require the user to specify the indices to keep manually.
Work is ongoing~\cite{peloton} to remedy this problem, but is not yet a production tool.
In the meantime, if we wish our translated datalog queries to run efficiently, the database must be provided with a list of indices to keep.

We tried a number of heuristics, including indexing in a global attribute ordering, indexing per query based on left-to-right joins, and just indexing all fields in order, and having the programmer reorder fields to boost performance.
None of these approaches worked in practice.
Both the global ordering and the left-to-right joins failed in large part because the query planner would choose to reorder the joins at runtime in multiple different ways.
The programmer manually ordering fields could find local optima, but because predicates are used in multiple ways, it too falls short.

The solution in use at the time this approach was switched away from was to annotate the program with an explicit set of indexes to keep.
We generated these indices by profiling the running program, and adding indices which would allow the query planner to avoid nested loops or full table scans where possible.

\subsubsection{Append-Only, High Write}
% Append only workload
One interesting aspect of a datalog system that the workload is entirely append-only other than retraction events, which are intended to be rare.
This knowledge is unused by the database in executing queries.
If it materializes a view to execute a query, and an underlying table is updated by an append, it will re-materialize the whole view, not perform any kind of incremental maintenance.

One of the expensive parts of many queries was insisting that it only return results which contained at least one \emph{new} fact - one which hadn't been returned in this query before.
That tables can only be appended to could enable the incremental maintenance of the join, allowing more efficient computation of the join, and retrieval of only the new data.

There are also some database schemas (such as the star schema) which become more possible in the absence of mutation or deletion.
\subsubsection{Query Planning}
Query planning, while of benefit to users who do not know all their SQL ahead of time, or whose tables remain in steady states, was the biggest issue with this approach.
Databases commonly use a component called a query planner to translate SQL statements into an internal representation (loops, merge joins, hash joins, index walks, etc) that they can concretely execute.
This component depends on a variety of information, including but not limited to:
\begin{itemize}
	\item Whether the statement was prepared
	\item If prepared, how many times it has been executed
	\item What indices are available
	\item Information from the statistics daemon
\end{itemize}
Other than examinining what indices are available, these conditions turn out to be highly anti-productive for a datalog workload.

The statistics daemon is designed with the assumption that there is a sort of ``steady state'' for a database, in which the relative sizes of the tables will remain similar.
This makes sense for usual customers of databases, but in our case, a large part of operation looks like heavy insert activity on a specific table.
As a result, the statistics daemon's information is generally woefully out of date.

We prepare virtually all statements, since we intend to execute them repeatedly and want to avoid time in the parser.
However, as of the time this system was developed, postgres would ossify the query plan as of the 5th time a prepared statement was executed.
This was done based on the assumptions that SQL connections do not live so long that the database changes a lot, so by the fifth time the query is run, the plan is unlikely to be improved, and performance will be increased by avoiding the planner entirely.
In practice, this means that any recursive rule (like one marking nodes as reachable, or performing a dataflow) will have suboptimal performance.
The rule executes five times, and during that time, the statistics daemon either has old out of date information, or even if it updates, information that the table it's reading out of is terribly small.
The query planner then makes bad decisions based on this, then sets them in stone.
As a result, indices sit there unused, and logarithmic operations are done linearly.

If the statements are not prepared, we incur parsing and planning overhead on every query.
While unfortunate, those costs were low in comparison to the troublesome queries.
The true problem with completely non-prepared statements is that the query planner would rapidly change strategies, meaning that which indices are needed would change at different points in execution.

Since in our case we have a fixed query set and a rapidly changing database, it would most likely make more sense to absorb the query planner into the compilation process somehow.
Postgres did not at the time of implementation have a way for a client to provide it with an explicit query plan short of building and providing a plugin which ran said plan as a function.

\subsubsection{Star Schema}
%TODO: note that this is equivalent to picking a good $I$ for our model, finding it, then swapping it back to Herbrand at the end
As alluded to earlier, one benefit of an append-only workload is that star schemas have lower overhead, as garbage collecting the child tables is not necessary.
Star schemas are normally used for ``data warehousing'', a sort of large scale database where an organization's data is all loaded into a single schema before being pulled out again into smaller databases for actual processing.
The idea is that most values are referenced rather than included directly in tables.
Warehousing personell are largely interested in the standardization of these values and the resulting compression.

In our case, a star schema is interesting both for reasons of compression, and for ease of indexing.
Indexing an IL instruction sequence\footnote{
IL instruction sequence refers to a lifted representation of an assembly instruction or sequence of assembly instructions into an "intermediate language" which as fewer, more orthogonal operations.
}, whether by hash or by ordering, is much slower than sorting by a tuple of integers.
We discovered this technique after the pivot to an in-memory database, so I have no observations of its performance, but I expect it would help.

\subsubsection{Large Objects}
With an external database, the use of large objects becomes nontrivially expensive.
If the database is local, and the bus between the program and the database is shared memory, this is not a major issue.
However, even over a local unix socket, repeated accesses to large objects can inhibit performance.

This shows up in practice when dealing with binary sections and segments during lifting.
If the lifting rule needs the segment, the architecture, and an offset into that segment to perform the lifting, this can incur several copies of the segment per instruction.
In my sample programs, most segments were between 300k and 600k bytes, causing this to incur a nontrivial cost.

The first solution, specific to this problem, was an all-at-once chunking of the segment.
We requested the segment from the database, then produced a 16-byte chunk (maximum length of an x86 instruction is 15 bytes) at every offset, and sent it back.
In the future, requests would access this chunked data rather than the original.
This resulted in a 16-17x blowup of the space to store the base binary, but as that paled in comparison to everything else it was not especially significant.

The second solution was to add another extension to datalog allowing some functions to exist as special external predicates to be run database side.
These all needed to be builtins, and while the approach was slightly more efficient, overall I no longer think the improvement warranted the complexity.

If I were to address this again today, I would use a star schema database side, and implement a cache client side for fetched star objects.

\subsection{Mycroft}
\label{sec:mycroft}
Mycroft is a row-oriented, single-threaded, in-memory datalog engine, taking into account the experiences of the initial implementation.
It operates as a macro which transforms datalog into Rust code, which can then be compiled into a running program.
In its current form, it addresses most, though not all, of the pain points encountered with Postgres.
The query planner is replaced by a single plan, generated at compile time, which parameterizes itself only on the size of the relevant tables at that moment.
This replacement also means that we know precisely what indices will be useful, and can generate them.
The join algorithm is aware of the incrementality of append only joins, and uses this to speed up requests for new results. 
As Mycroft is in process and in memory, large objects are not a problem.
They are returned as read-only references to the existing strucure, and can be operated on that way.
The implementation is available at \url{https://github.com/maurer/mycroft}, and as a crate on \texttt{crates.io} for direct inclusion in rust projects.

\subsubsection{Typed Storage}
Rather than store data values directly in rows as was done in the Postgresql-based implementation, here we keep a separate deduplication table for each type of data on which we operate.
This allows us to efficiently map back and forth between values, which the callbacks need to consume, and integer keys, which are convenient for indexing and join algorithms. 
This is reminiscent of the star schemas discussed before.
As our system is mostly-append (other than retractions due to circumscription) we design this as an insert-only structure.
An additional benefit, more relevant here than with Postgres, is that this greatly reduces our memory footprint.

At compile time, each type present in one or more predicates has a modified robin-hood hash table declared for it.
This table has two pieces: a vector backing which stores the actual data, and a vector of hash/key pairs.
There are two operations this table needs to support: acquiring the key for an object, whether or not it's present already, and acquiring an object from its key.
Finding the key for an object is accomplished by using a lookup on the hashtable portion of the structure, inserting into both the table and the vector of data in the event of a lookup failure.
Finding the object for a key (the more common case) works by indexing into the vector.

The only principal difference between this and a simpler design (a standard hash table mapping from the value to the key, and a vector mapping from a key to a value) is that it stores the data only once, and without any indirection.
While this may not sound like much, this gave a modest 23\% time performance boost over the standard library implementation in time, and approximately halved space on an earlier version of the use-after-free detector.
The closest approach still using the standard data structure would have been to use a smart pointer to share data between the data structures, or a hashtable of hashes.
The smart pointer caused trouble with the interfaces, and hashing twice incurred a performance penalty, so we used this custom hastable design for deduplication and unique key assignment.

\subsubsection{Aggregation}
As aggregation is described at the predicate level, we can implement it directly on the tuple storage.
Tuple storage is structured as a map from the tuple of non-aggregate fields (reordered to the front) to a tuple of aggregate fields.
These aggregate fields are represented by a triple of the value-keys to be aggregated, a current aggregate value, and an index indicating how many of the value-keys are aggregated in the cached value.
This allows for a lazily updated computation of the meet.

When a tuple is inserted into the store, if a value with the same non-aggregate fields is present, the value-key list is extended, but the aggregate is left alone.
If it is not present, we initialize the aggregate value with the value of the key in that slot, fill in the key in the keys-to-be-aggregated, and set the index to 1.
When retrieving a tuple, we check whether the index is equal to the length of the comprising keys.
If it is not, we start the iteration at the index, and perform iterative meets until the aggregate is up to date.
We then return the tuple, extended by the aggregate fields and reordered.

\subsubsection{Join Computation}
Datalog computation is join-heavy, and as a result attempting to compute the join naively can lead to disasterous execution times.
There are a variety of existing join approaches.
% Cite postgres?
RDBMSes tend to favor straightforward strategies, such as nested looping, hash join, and merge join.
Merge joins require a relevant index, but generally perform substantially better unless tables are extremely small.
Hash joins operate by creating an intermediate data structure of one of the tables which is indexed by the hash of the joined values.

However, for high-arity join patterns, better algorithms exist, usually formulated as ``worst-case join'' algorithms.
Ngo showed~\cite{nprr} that it is possible to develop join algorithms which are optimal even under these conditions.
This algorithm is rather complex, and is intended for theoretical results rather than actual implementation.
However, LogicBlox~\cite{logicblox} developed an algorithm known as Leapfrog Triejoin~\cite{lftj} which achieves these same bounds while remaining practically implementable over traditional indices.
Unfortunately, this algorithm is patented, and so could not be used.
This indicates a potential for future implementations to derive a novel approach from the AGM~\cite{agm} bound or Ngo's~\cite{nprr} approach, but developing such an algorithm is beyond the scope of this thesis.

In Mycroft, we used a simultaneous merge join ordered from smallest table to largest table.
An index is selected for each table which walks it in unification argument order, with constant arguments being sorted to the front.
The first index is advanced to the first tuple where all the constant arguments match.
This is made easier by the use of integer-only tuples, as the non-constant arguments can be represented as 0 in a query to the index.
Then, candidate variable bindings are made to the unification terms (if possible) and the next table is considered.
When on the last table, if a candidate set of assignments to the unification terms can be completed, it is emitted and the index advanced by one step.
If the index cannot advance or the index fails to unify with earlier tables, we know that no further result is possible, and go back one table, and continue.
This approach keeps around only a small amount of additional state, linear in the number of clauses in the query, as it is returning results.

However, due to our need for \emph{incremental} results, we can improve this mechanism substantially.
Rather than computing the entire join at once, we split it into subjoins, one for each clause in the query.
We have a separate, much smaller index for ``new'' facts in referenced predicates, requested by the query at database initialization time.
We perform a subjoin with each predicate's large index swapped for this small index to get exactly those results which we would receive that we did not before, then chain them for a result.
The small indexes are emptied during this operation, so they will not yield the same results again.

As an example, consider evaluating the query $A(x, y) \& B(y, z) \& C(z, x)$ for incremental results.
The first time it is evaluated, we perform a full join, ignoring the subjoin strategy - it would be equivalent to performing the full join 3 times.
Then, we insert two facts into $A$, and one into $C$.
Running the query again, we perform three subjoins, one on $A', B, C$, one on $A, B', C$, and one on $A, B, C'$.
In our join algorithm above, remember that we sorted the smallest table to the left.
As a result, the join with $B'$ immediately terminates, yielding no results.
For the join with $C'$, it essentially acts as a join on $A$ and $B$ only, with a constant restriction.
The join with $A'$ is similar, but the $A'$ portion of the join yields two facts, so it essentialy runs two constant constrained joins of $B$ and $C$.

\subsubsection{Provenance Tracking}
In order to later manage circumscription, or to allow a human to trace the reasoning of a program, we need to keep track of where facts come from.
To do this, in conjunction with each tuple we store a list of possible justifications.
A justification is composed of the ID of a rule, and the IDs of the facts used to match the body clause of that rule.
An aggregation is represented simply as the list of fact IDs aggregated for the match.
A map is additionally maintained from fact IDs to justifications which contain them.

\subsubsection{Circumscription, Call/CC, and Retraction}
Implementing circumscription essentially involves monitoring accessed aggregations to see if they would change, and responding with a retraction.
The previous description of aggregates does not easily allow for this.
A tuple insertion does not know if something has depended on this aggregation's completeness, and if so what.
To deal with this, if a tuple is circumscriptively fetched, we replace the list of merged keys in the aggregate field with a newly minted aggregate ID.
Three maps are maintained for aggregate IDs:
\begin{itemize}
	\item Aggregate ID to comprising Fact IDs
	\item Aggregate ID to dependent justifications
	\item Fact IDs to Aggregate IDs they comprise
\end{itemize}

If a tuple insertion occurs and would need to update an aggregate represented by an aggregate ID, that aggregate ID is retracted.
The retraction code acts as a worklist, initially populated by the broken aggregation.
First, it removes any justifications broken by the current retracted item.
Then, it retracts (by adding to the worklist) any facts which now lack justification.
If the current retracted item is a fact, it also retracts any aggregate IDs which now have one fewer fact.

In the special case where the tuple just inserted was also retracted, we replace its justification with one referencing the members of its now broken aggregate ID.
This ensures that while this justification no longer cares about the expansion of the circumscription, it will still be properly retracted if one of the facts in the original aggregate becomes invalid.

%TODO explain how we deal with cyclically supported facts
% e.g. \neg B -> A; A -> A; discover B, how do we ensure we retract A?
% Also explain how this works in the presence of circumscription.

\subsubsection{Future Work}
There is plenty of room for improvement in the concrete implementation of the language engine.

Currently, we keep more indices than are strictly necessary.
Even with our current join strategy, the count of indices kept scould be reduced through a mechanism to match attribute ordering between queries more frequently.
With a more modern join like tetris join~\cite{tetris} it could even be possible to keep a single index per predicate.

Results of some subjoins get used repeatedly, and can be known not to change through topological sorting.
Currently, this is exploited through manual tabling - the creation of dummy predicates to keep the completed join as a realized structure.
However, it should be possible to generate these temporary structures automatically in some cases.

Pivoting indices from a simple in-memory btree to a MVCC\footnote{
	MVCC stands for Multiple Version Concurrency Control.
	MVCC trees are common in database design because they map well to the block-at-a-time disk update structure and because they allow for multiple transactions to act on the same index in a way that makes it clear if the index was invalidated while using it.
	They accomplish this by retaining any portion of the tree which is being accessed by some transaction, and garbage collecting as threads leave.
	This results in "multiple versions" of the tree being accessible simultaneously in order to deal with concurrency contention, thus the name.
}-style structure would allow multiple worker threads to be evaluating rules at the same time.
As modern systems generally have additional cores, this should lead to performance improvements overall (though degradation in bottlneck phases).
This approach also meshes well with optionally backing some data structures with disk due to either large size or low traffic.
Many MVCC trees are already designed as on-disk data structures due to their use in traditional RDBMS systems.
Allowing some data to reside on disk would increase the maximum size of analysis the system could perform on a single binary, or allow for easier multi-binary analysis.

In an ideal world, this system could even be distributed.
Other than circumscription and the decision to terminate, every component of this system can operate safely with a partial knowledge of the database.
As a result, it seems plausible that with appropriate heuristics for shuffling and synchronization around circumscription, this language could be well suited for distributed execution.

