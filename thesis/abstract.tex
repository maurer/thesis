%Vulnerability detection in compiled code can be more effectively performed using a logic-language based framework which supports non-stratified negation and co-dependent analyses.
%Compiled code needs to be analyzable
We need program analysis tools and techniques that address the unique
problems of executable code software analysis.
%In order to perform security audits of executable programs and
%libraries,  and develop attacks,
%This is difficult because binaries are turing complete, and even "well behaved" code does crazy shit
When code is compiled, the compiler strips away source code
information, such as control flow, types, and variable locations.  As
a result, program analyses of compiled code faces a number of unique
challenges. % as a direct consequence of dropping information
%during the compilation process.
%General statement about how we can help?
Previous work has proposed techniques to perform single analysis, such
as control flow recovery, variable identification, and value analysis.
However, previous work has not focused on the \emph{co-dependencies}
between these individual analysis. For example, value set analysis
requires control flow analysis, which in turn requires value set
analysis. We need techniques that can handle such co-dependencies in
order to effectively check higher-level
security properties such as use-after-free bugs. % , nor proposed a
% holistic platform for integrating new co-dependent analyses.
% %Previous Work
%  However, each of these individual analysis are co-dependent
% in unique ways at the binary level.  These co-dependent analyses must be
% able to iteratively call each other to refine their results.

% benefits from type analysis, which in turn depends on control flow
% analysis. Control flow analysis

% For example, type recovery depends on dataflow information, which in
% turn depends on control flow which is an independent analysis.

% Recovering non-trivial control flow
% requires value analysis to resolve jump targets.  One variety of value
% analysis (alias analysis) is sometimes done using type information.

% However, previous work tended to focus on one or two
% specific questions at a time rather than holistic analysis.

In this thesis we propose using a logic-language based approach to
encode co-dependent analysis, and build a tool called Holmes based on
our approach to demonstrate our ideas on real code. We demonstrate
several novel techniques for extending Datalog semantics to
efficiently and effectively reason about real binary code. For
example, our approach allows one to elegantly encode the co-dependence
between value-set and control-flow analysis described above. We
demonstrate {\sysname} and our approach by encoding several real-world
co-dependent analysis, and showing that {\sysname} can effectively be
used to check higher-level properties such as use-after-free on real
executable code.

Random ideas:
1. Put Bitr into section 2 as a motivating example.  E.g., section 1
is a complete overview. Then we use section 2 to provide motivating
examples. The section 3 describes holmes.


%
%effectiveness concretely by implementing a
% use-after-free detector over binaries.  We further present a binary
% type recovery system which despite not being written with \sysname\ in
% mind, is still structured such that it would be amenable to
% integration.


% Holmes is a well-defined system that extends datalog with appropriate
% negation and aggregation

% C
% o-dependent analyses are more important when traditional abstractions
% designed to allow for modular reasoning have been removed.  Type
% recovery depends on dataflow information, which in turn depends on
% control flow.  Recovering non-trivial control flow requires value
% analysis to resolve jump targets.  One variety of value analysis
% (alias analysis) is sometimes done using type information.  While
% co-dependency is most prevalent in control flow analysis, it finds its
% way into nearly every part of binary analysis.

% In this thesis we have implemented \sysname, a datalog dialect suited to this task of knitting together co-dependent analyses, and formally defined its operation.
