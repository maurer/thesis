\paragraph{Multithreading tolerance.}
\label{tach:sec:multi}

Up until now, we have considered only programs using one
thread. However, many modern programs use multiple threads in their
normal operation. For example, \texttt{curl} in \S~\ref{tach:sec:eval} uses them during DNS
lookup.

To deal with threads, \tachyon employs techniques inspired by the field of
deterministic multithreading (DMT)\cite{coredet, determinator,
  dthreads}.  A DMT mechanism is one that makes a program insensitive
to the scheduler as an input. That is, given the same inputs other
than kernel scheduler action, it will yield the same outputs. 

In \tachyon, we enforce an ordering over system call events.
We differentiate from a mismatched system call in need of rewriting and a thread
being early or late by choosing to block the thread if the thread IDs on the syscalls don't match,
and invoke the rewrite engine if they do match, but the system calls don't.
This forms a looser notion of consistency than is used in regular DMT, but is sufficient for our purposes.
However, applying a real DMT system in addition to our techniques
would likely yield an even more robust treatment of threading able to deal with shared memory
data transfers and other such intricacies.
